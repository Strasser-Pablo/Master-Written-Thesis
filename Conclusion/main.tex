\chapter{Conclusion}

In conclusion we were able to have a ``simple'' scheme to solve free surface incompressible 
Navier-Stokes equations.
For this we have used the following element:
\begin{itemize}
 \item A spatial discretization on a staggered grid. And definition of differential operator on it.
 \item The reduction of the original problem to an ODE with a projected function.
 This need to solve a linear system at every time step.
 \item A topological marker particle and interpolation method on the marker.
 \item An extrapolation operator that need to ensure boundary conditions.
\end{itemize}

We were not able to launch 3d run and are aware of the high numerical cost of direct Navier-Stokes
methods for small viscosity problem, but juged that it was the better method to begin with.
\section{Possible improvement}

The following theoretical improvement are possible:
\begin{itemize}
 \item Use higher order spatial discretization.
 \item Use other topological marker instead of particle like level set or boundary surface representation.
 \item Have better discretization at free surface boundary.
 \item Have a variant of Runge-Kutta method that are correct if we have Dirac delta in the function 
 or use smooth topological change.
\end{itemize}

The following code improvement are possible:
\begin{itemize}
 \item Have a linear solver which can use the fact that topology doesn't change much between iteration
 to be faster in average.
 \item Make the code run on parallel cpu and gpu.
\end{itemize}

\section{Software used}

The code was written primary in \textbf{C++11}.
The final version use the following library:
\begin{itemize}
 \item Umfpack as linear solver.
 \item VTK for output in the vtk format used in the visualization software Paraview.
\end{itemize}

Previous version used:
\begin{itemize}
 \item Boost library for python binding and externalization.
 \item Pyamg in python for multigrid.
 \item VianaCl for multigrid.
 \item Lusol in fortran for sparse Lu updating.
 \item Spooles as linear solver.
\end{itemize}

For postprocessing and generate this document the following software were used:
\begin{itemize}
 \item Lua\LaTeX \ as \LaTeX \ program with some lua script to generate picture.
 \item Pgfplots and tikz for picture and graphics.
 \item Matplotlib in python to generate picture and graphics.
 \item Paraview to visualize the data.
\end{itemize}

For debuging of the \textbf{C++11} code in addition to GDB, Valgrind was used to track memory
related error and memory leak.

\section{Thanks}

I am very thankfull to:
\begin{description}
 \item[Prof. Martin Gander:] for the supervising of the work and all my numerical analysis knowledge.
 \item[Dr. Félix Kwok:] for the supervising of the work and contact in pratical numerical analysis.
 \item[Prof. Peter Wittwer:] for the Mathematics and Physics courses in my Physics studies.
 \item[My family: Reto, Norma, Roland, Tamara:] for supporting me and all the fun together.
\end{description}

