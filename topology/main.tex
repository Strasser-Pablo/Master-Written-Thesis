\chapter{Variable Topology}

\section{Introduction}
In the previous chapter all numerical scheme were done on the same domain.
We will here see the change that we need to make if we want to have variable topology.


We will need to take care of two additional thing:
\begin{enumerate}
\item Know the speed in cell that weren't in the domain but are in the new time step.
\item Know the position of the different boundary.
\end{enumerate}

We are mainly interested in mobile boundary consisting in free surface boundary because it will be the only
boundary that move depending on the solution.

The motion of the domain and the boundary is given by:
\begin{equation}
\dot{\vect{x}}=\vect{v}
\end{equation}

Where $\vect{v}$ is the speed at the given position.

To discretize this equation, we will use particle  placed in every cell. A cell where a particle lie is in the domain.
A cell without particle is out of the domain.

But particle move with the speed given at there position, witch is not on a grid point.

Because of this, we need to interpolate the speed at the given point.

When a particle move to a new cell, we need to know the initial speed at this position.
We analytically have boundary condition for speed at boundary.
We need to extrapolate this boundary condition one cell away.

\section{Analytical equation}

We can write a none discretized particle method equation as:

\begin{align*}
	\dot{v}&=\left(\eye + P(D(\vect{x}))\right)f(\vect{v},\vect{x})\\
	\dot{x}&=v
\end{align*}

Where $x$ are a continuous representation of the space with particle.
The notation $P(\vect{x})$ say that we consider the projection operator with boundary condition given by $\vect{x}$.

We need to had to this boundary condition that are given by a linear differential operator $A$.
\begin{equation}
	A\vect{v}=0
\end{equation}
$A$ contain derivative with respect to space but not respect to time.

When we discretize, we will need to interpolate because we need speed at position we don't know (for a staggered grid we have):
\begin{align*}
	\left.\dot{\vect{v}}\right|_{\in D_{v}(D(\vect{x}))}&=\left(\eye+P(D(\vect{x}))\right)\left.f(E(\vect{v},D_v(D(\vect{x}))),D_{v}(D(\vect{x})))\right|_{\in D_{v}(D(\vect{x}))}\\
	\dot{x}&=I\left(\vect{x},E(\vect{v},D_v(D(\vect{x})))\right)\\
	\left.\vect{v}\right|_{\notin D_{v}(D(\vect{x}))}&=\left.E(\vect{v},D_v(D(\vect{x})))\right|_{\notin D_{v}(D(\vect{x}))}
\end{align*}

For mathematical simplicity we will take that $\vect{v}$ consist of a speed defined everywhere, 
but in the computation only speed that are accessed in the time step is needed (only speed in the domain and in it's neighbour).

$D(\vect{x})$ indicate the domain in witch at least a particle is found. This consist at the same time to the topology for pressure (Pressure are cell centered).
$D_{v}$ is a function depending of the topology given in general by $D(\vect{x})$ and indicate witch speed component are in the domain.
$f$ is a function that take as argument the topology for pressure given by $D(\vect{x})$ and the topology for speed given by $D_v(D(\vect{x}))$ and the speed everywhere.
$E$ is the extrapolation operator witch take as argument the speed everywhere and return a speed everywhere,
but where the speed in the speed domain is not changed.
$I$ is an interpolation operator witch from the speed everywhere find the speed at a given position.

The use of two topology is needed because in a staggered grid the position of speed and pressure is not the same.

$D_{v}$ need to be good chosen to avoid pathological case.
For example for a cell with only one cell. We will normally consider all speed as boundary. But with only boundary we have no motion.
For this reason in this case, we consider all boundary cell as in the domain.

We will now rewrite this as a differential equation from time $t_0$ to $t_0+dt$:
\begin{align*}
	\left.\dot{\vect{v}}\right|_{\in D_{v}(D(\vect{x}))}&=\left(\eye+P(D(\vect{x}))\right)\left.f(E(\vect{v},D_v(D(\vect{x}))),D_{v}(D(\vect{x})))\right|_{\in D_{v}(D(\vect{x}))}\\
	\left.\vect{v_{t_0}}\right|_{\notin D_{v_{t_0}}(D(\vect{x}))}&=\left.E(\vect{v_{t_0}},D_v(D(\vect{x})))\right|_{\notin D_{v}(D(\vect{x}))}\\
	\dot{x}&=I\left(\vect{x},E(\vect{v},D_v(D(\vect{x})))\right)\\
\end{align*}

\subsection{Splitting}

A simple but of only order $1$ (in general) mean to solve this equation, is to consider two different problem.
The first advance speed for a given topology, then advance topology from the given speed.

The first step is to solve:
\begin{align}
\left.\dot{\vect{v}}\right|_{\in D_{v}(D(\vect{x}))}&=\left(\eye+P(D(\vect{x}))\right)\left.f(E(\vect{v},D_v(D(\vect{x}))),D_{v}(D(\vect{x})))\right|_{\in D_{v}(D(\vect{x}))}\\
	\left.\vect{v_{t_0}}\right|_{\notin D_{v_{t_0}}(D(\vect{x}))}&=\left.E(\vect{v_{t_0}},D_v(D(\vect{x})))\right|_{\notin D_{v}(D(\vect{x}))}
\end{align}

Witch can be done like in the fixed topology case with a Runge-Kutta method. The second equation indicate the boundary condition.
Numerically we can calculate the speed at boundary at every evaluation of $f$.

We then move the particle with the calculated speed.
\begin{equation}
	\dot{x}=I\left(\vect{x},E(\vect{v},D_v(D(\vect{x})))\right)\\
\end{equation}

This method is only of order 1. We can see it from the following example.
Consider only one particle. With as initial condition 0 speed without viscosity.
The extrapolation operator in this case will copy the speed vertically,
the speed will be constant and the only none zero component is in the $y$ direction.
The speed will be given by the integration of the gravity force, the convective term is 0 because speed is constant.
The speed will be calculated with an accuracy of order $4$.
But the position will be calculated from the advance of a constant speed.

In final, our speed will be correct at order $4$ but position will be wrong.

This error doesn't depend on the precision of integration for the speed or position problem.
It come from the splitting.

\subsection{One differential equation}

We will now show how we can rewrite this differential equation in only one differential equation.
For this, we need an expression for the derivative of the boundary condition.
When we have it, we only have to use a Runge-Kutta method.


We will now only be interested in the case where $E$ is a linear operator white the two following constraint:
\begin{itemize}
	\item It's effect for speed in the domain (for speed) is the identity.
	\item It's effect for speed out of the domain (for speed) depend only on speed of the domain (for speed).
\end{itemize}

The domain for speed for $E$ is noted $E(D_v(D(\vect{x})))$.

In matrix notation it tell that some block need to be $0$ or $\eye$.
If we consider that indice from $1$ to $n_d$ are in the domain(for speed) and $n_d+1$ to $n$ are out of the domain (for speed), the constraint for a line of the matrix is,
using as notation the : as in matlab in indicating by an indice it's position in the vector:
\begin{align}
	\intertext{For $i<n_d+1$}
	E_{i :}&=\left( 0 \ldots 1_{i} \ldots 0\right)\\
	\intertext{For $i>n_d$}
	E_{i :}&=\left( e_{1}\ldots e_{n_d} 0\ldots 0\right)\\
\end{align}

Taking the derivative of $E(D_v(D(\vect{x})))\vect{v}$ has two term, the first give:
\begin{equation}
	E(D_v(D(\vect{x})))\dot{\vect{v}}
\end{equation}

Witch use $\dot{\vect{v}}$ but need it only for in the domain because outside the given column of the $E$ matrix is zero.

The second therm is the derivative of the matrix $E$ with respect of change of time. Witch will lead to the derivative with respect of change of topology.
This will give a delta of dirac.

The reason of this is because of the discontinuity in time of the boundary speed, when we change topology.

We will neglect this dirac delta. The equation is then:
\begin{align}
	\left.\dot{\vect{v}}\right|_{\in D_{v}(D(\vect{x}))}&=\left(\eye+P(D(\vect{x}))\right)\left.f(E(D_v(D(\vect{x})))\vect{v},D_{v}(D(\vect{x})))\right|_{\in D_{v}(D(\vect{x}))}\\
	\left.\dot{\vect{v}}\right|_{\notin D_{v}(D(\vect{x}))}&=E(D_v(D(\vect{x})))\left(\eye+P(D(\vect{x}))\right)\left.f(E(D_v(D(\vect{x})))\vect{v},D_{v}(D(\vect{x})))\right|_{\in D_{v}(D(\vect{x}))}\\
	\left.\vect{v_{t_0}}\right|_{\notin D_{v_{t_0}}(D(\vect{x}))}&=\left.\left(E(D_v(D(\vect{x})))\left.\vect{v_{t_0}}\right|_{\in D_{v}(D(\vect{x}))}\right)\right|_{\notin D_{v}(D(\vect{x}))}\\
	\dot{x}&=I\left(\vect{x},E(D_v(D(\vect{x})))\vect{v}\right)\\
\end{align}
Where we have used the notation that vector can be extended with 0 when needed because the extension is not needed because the good column are zero.

Witch can be rewritten as:

\begin{align}
	\left.\dot{\vect{v}}\right|&=\left(\eye_{\in D_{v}(D(\vect{x}))}+E(D_v(D(\vect{x})))\right)\left(\eye+P(D(\vect{x}))\right)\left.f(E(D_v(D(\vect{x})))\vect{v},D_{v}(D(\vect{x})))\right|_{\in D_{v}(D(\vect{x}))}\\
	\left.\vect{v_{t_0}}\right|_{\notin D_{v_{t_0}}(D(\vect{x}))}&=\left.\left(E(D_v(D(\vect{x})))\left.\vect{v_{t_0}}\right|_{\in D_{v}(D(\vect{x}))}\right)\right|_{\notin D_{v}(D(\vect{x}))}\\
	\dot{x}&=I\left(\vect{x},E(D_v(D(\vect{x})))\vect{v}\right)\\
\end{align}

This equation can be integrated with a Runge-Kutta method.
For having a maximal precision, the boundary speed is taken from the extrapolation at every complete Runge-Kutta step.

In the argument of $f$ we have used an extrapolation of speed witch don't depend of the input in boundary speed.
The integrated boundary speed is only used in the case where a cell become in the domain in a time step.

The neglected dirac delta will only have an effect on a boundary speed component in the case where in a Runge-Kutta step the following happen.
A neighbor cell(cell witch $E$ matrix as component with) fill. This will change the value of boundary speed with a jump (this jump will be neglected).
Then the speed need to be fill between the Runge-Kutta step so that the wrong value is conserved and not erased.

This effect will change the precision when it happen from order $k$ to order 1.
And this error will only be on the boundary condition being not exactly respected for a short time.
The force will be integrated correctly.

The direct filling of a cell don't pose problem because we don't have discontinuity in speed, because in the first region we force speed,
in the second the derivative.

The jump is only caused because we force speed differently for the two region.


Taking the same problem with one particle, given by. A particle with 0 initial speed, without viscosity.
The speed will be constant. So only the gravity force act and the extrapolation operator will copy speed vertically.
The result will then be equivalent to the integration of the purely particle problem where the speed is change is calculated at particle position.
The order of the method is then $k$.

\section{Extrapolation}

One important point is how we enforce the boundary condition in discretize grid.

In the boundary we have (for 3d case):
\begin{align}
	\sum_{i,j}\sigma_{ij}n_{i}n_{j}&=0\\
	\sum_{i,j}\sigma_{ij}t^{1}_{i}n_{j}&=0\\
	\sum_{i,j}\sigma_{ij}t^{2}_{i}n_{j}&=0\\
\end{align}

$n$ is the normal vector of the surface.
$t^{1}$ and $t^{2}$ are none colinear tangent vector.

In 2d we only have one tangent vector.

$\sigma_{ij}$ is given by:
\begin{equation}
	\sigma_{ij}=-p \delta_{ij}+\nu\left(\frac{\partial v_{i}}{\partial x_{j}}+\frac{\partial v_{j}}{\partial x_{i}}\right)
\end{equation}

The first equation can be used to find boundary condition for $p$.
With:
\begin{equation}
	p=O(\nu)
\end{equation}

$\nu$ is small for water. This justify the often used boundary condition of 0.

The other equation only depend of speed, because $n$ is orthogonal to $t^{1}$ and $t^{2}$.
The value of $\nu$ is not relevant for the condition (a global none zero constant).
The constraint is linear. With sufficiently equation and know value we can find an unique solution.

We now will treat the different possible case in 2d. This will then be generalized in 3d.

\subsection{Plane}

The first case is a surface witch is a plane following one of the cell boundary.

Without loss of generality we consider the plane in the $x$ direction.

The normal and tangent vector are then given by:
\begin{align}
	\vect{n}&=\begin{pmatrix}
	         	0\\
	         	1
	         \end{pmatrix}\\
	\vect{t}&=\begin{pmatrix}
	          	1\\
	          	0
	          \end{pmatrix}
\end{align}

The equation are then given by:
\begin{align}
	\sum_{i,j}\sigma_{ij}t_{i}n_{j}&=0\\
	\sigma_{12}&=0\\
	-p \delta_{12}+\nu\left(\frac{\partial v_{1}}{\partial x_{2}}+\frac{\partial v_{2}}{\partial x_{1}}\right)&=0\\
	\frac{\partial v_{1}}{\partial x_{2}}+\frac{\partial v_{2}}{\partial x_{1}}&=0\\
\end{align}

We discretize this equation at point $i+\frac{1}{2},j+\frac{1}{2}$:
\begin{equation}
\label{var:extr:droitCont}
	\frac{v^{1}_{i+\frac{1}{2},j+1}-v^{1}_{i+\frac{1}{2},j}}{\Delta x_{2}}+\frac{v^{2}_{i+1,j+\frac{1}{2}}-v^{2}_{i,j+\frac{1}{2}}}{\Delta x_{1}}=0
\end{equation}

Where $\Delta x_{1}$ and $\Delta x_{2}$ is the spacing in $x$ and $y$.
We have one equation for 3 unknows, only $v^{1}_{i+\frac{1}{2},j}$ is know.
But we can add two equation using the divergence free condition in the fluid for cell $i,j$ and $i+1,j$.

This give us:
\begin{align}
\label{var:extr:droitContA}
	\frac{v^{1}_{i+\frac{1}{2},j}-v^{1}_{i-\frac{1}{2}},j}{\Delta x_{1}}+\frac{v^{2}_{i,j+\frac{1}{2}}-v^{2}_{i,j-\frac{1}{2}}}{\Delta x_2}&=0\\
\label{var:extr:droitContB}
	v^{2}_{i,j+\frac{1}{2}}&=v^{2}_{i,j-\frac{1}{2}}-\frac{v^{1}_{i+\frac{1}{2},j}-v^{1}_{i-\frac{1}{2},j}}{\Delta x_{1}}\Delta x_{2}\\
\label{var:extr:droitContC}
	\frac{v^{1}_{i+\frac{3}{2},j}-v^{1}_{i+\frac{1}{2}},j}{\Delta x_{1}}+\frac{v^{2}_{i+1,j+\frac{1}{2}}-v^{2}_{i,j-\frac{1}{2}}}{\Delta x_2}&=0\\
\label{var:extr:droitContD}
	v^{2}_{i+1,j+\frac{1}{2}}&=v^{2}_{i+1,j-\frac{1}{2}}-\frac{v^{1}_{i+\frac{3}{2},j}-v^{1}_{i+\frac{1}{2},j}}{\Delta x_{1}}\Delta x_{2}
\end{align}

We substitute the two equation \ref{var:extr:droitContB} and \ref{var:extr:droitContD} in equation \ref{var:extr:droitCont}.
Witch will now only depend on one unknows $v^{1}_{i+\frac{1}{2},j+1}$ witch depend on quantity that can be found using the speed in the domain.

We can only do the thing that we have done only if we have a depth of minimal two cell.
If we have only one cell, we have no $y$ speed in the domain in calculation.
For this case, consider the speed $y$ at boundary as in the domain.

\subsection{\unit{45}{\degree} Plane}

If we have a cell in domain with two adjacent cell witch are not in the domain.
We take as normal a \unit{45}{\degree} Plane.

The normal and tangent vector are then:
\begin{align}
	\vect{n}&=\begin{pmatrix}
	\frac{\sqrt{2}}{2}\\
	\frac{\sqrt{2}}{2}
	  \end{pmatrix}\\
	\vect{t}&=\begin{pmatrix}
	          	\frac{\sqrt{2}}{2}\\
	          	-\frac{\sqrt{2}}{2}
	          \end{pmatrix}
\end{align}

The constraint is given by:
\begin{align}
	\sum_{ij}\sigma_{ij}n_{i}t_{j}&=0
	\sigma_{11}n_{1}t_{1}+\sigma_{22}n_{2}t_{2}+\sigma_{12}(n_{1}t_{2}+n_{2}t_{1})&=0\\
	2\frac{\partial v_{1}}{\partial x_{1}}\frac{1}{2}-2\frac{\partial v_{2}}{\partial x_{2}}\frac{1}{2}&=0\\
	\frac{\partial v_{1}}{\partial x_{1}}-\frac{\partial v_{2}}{\partial x_{2}}&=0
\end{align}

We discretize at cell center $i,j$ witch give:
\begin{equation}
	\frac{v^{1}_{i+\frac{1}{2},j}-v^{1}_{i-\frac{1}{2},j}}{\Delta x_{1}}-\frac{v^{2}_{i,j+\frac{1}{2}}-v^{2}_{i,j-\frac{1}{2}}}{\Delta x_{2}}=0
\end{equation}

We have two unknows $v^{1}_{i+\frac{1}{2},j}$ and $v^{2}_{i,j+\frac{1}{2}}$.

We add an additional equation using the divergence free condition at cell $i,j$:
\begin{equation}
	\frac{v^{1}_{i+\frac{1}{2},j}-v^{1}_{i-\frac{1}{2}},j}{\Delta x_{1}}+\frac{v^{2}_{i,j+\frac{1}{2}}-v^{2}_{i,j-\frac{1}{2}}}{\Delta x_2}=0
\end{equation}

For this too work, we need to have the left and below boundary speed to be know.
The boundary need to be like that because in the contrary we will have no data in the $x$ or $y$ direction.

\subsection{Continuation}

We now have the extrapolation method for speed directly at the boundary.
We can generate the other case in ``filling'' the cell and using the same condition recursively.

\subsection{Speed domain}

We have noted in the previous section that we allow speed at boundary to be in the domain, if the contrary will make a domain without speed in domain
horizontally or vertically.

\section{Interpolation}

For interpolation, we don't have constraint on the sheme. A simple one can be for example $n$-linear interpolation.





