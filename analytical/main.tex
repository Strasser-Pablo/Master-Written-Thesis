\chapter{Analytical}
\minitoc
\section{Introduction}
In this chapter, i will discuss the analytical problem that we want to solve.
The other numerical chapter will consist of a numerical sheme to solve the problem given in this chapter.

\section{Variable and Parameter}

The problem will consist to find a function of space and time respecting some condition and equation.

\subsection{Space and Time}

The space consist of a vector in fixed cartesian coordinate with dimension typically 2 or 3 (Navier-Stokes on 1d is trivial).
Space is noted $\vect{x}$.
The time consist of a indice indicating the evolution of the function in time.
Time is noted $t$.

\subsection{Domain}

A domain is the spatial region where a field is defined. This consist physically in the spatial region where water lay.
Boundary condition need to be provided.

The Domain can vary with respect of time and can vary with respect of the solution at previous time.

The Domain don't need to be connected. But can be decomposed in union of connected Subdomain.

\begin{example}
 If we take a drop of water falling in a glass of water. The domain consist of the region where the water lie.
 At the beginning this will consist of domain consisting of two connected subdomain (the drop and the glass)
 that after a given time will merge in one domain.
\end{example}

The Domain is generally noted $\Omega$ or $\Omega(t)$ to emphasis that it can be variable with respect of time.

\subsection{Field}

A Field consist of a function with respect of space in the domain and time in the temporal domain.
\begin{equation}
 F(x,t)\qquad \text{With $x\in\Omega(t)$ and $t_0\leq t \leq t_f$}
\end{equation}

\subsubsection{Vector Field}
A vector field is a field that is a vector.

We only have one Vector Field named speed and noted $\vect{v}$.

The speed physically represent the speed of water at a given position.
\subsubsection{Scalar Field}
A scalar field is a field that is a scalar.

We only have one Scalar Field named pressure and noted $p$.

The pressure physically represent an internal force per unit area between water particle.

\subsection{Parameter}

The following parameter are used in this work.

\subsubsection{Viscosity}
The viscosity represent how much water particle stick together.

Higher is the viscosity less abrupt speed change is possible.

The viscosity is noted $\mu$. We use allot $\nu$ that is the viscosity divided by the density.
\begin{equation}
 \nu=\frac{\mu}{\rho}
\end{equation}

\subsubsection{Density}

The density is a mass per unit volume.

Density play the same role as mass in Newton's mechanic.

In the incompressible Navier-Stokes equation density is more a constant depending
of the domain because incompressibility tell us that density doesn't change.

The density is noted $\rho$.

\subsubsection{Force}

The force is a user defined vector field that represent the external force applied to the water.
It can possibly depend on the solution for previous time.

The force is noted $\vect{F}$

\section{Navier Stokes}

\subsection{Fixed Eulerian}
\label{analytical:fixe_eulerian}
The incompressible Navier-Stokes equation are given by:
\begin{subequations}
\begin{align}
\label{analytical:navierStokesA}
\vect{\nabla} \cdot \vect{v}(\vect{x} ,t)&=0\\
\label{analytical:navierStokesB}
\partial_t \vect{v}(\vect{x} ,t)+(\vect{v}(\vect{x},t)\cdot\vect{\nabla} ) \vect{v}(\vect{x} ,t)&=-\frac{\vect{\nabla} p(\vect{x},t)}{\rho(\vect{x},t)}+\frac{\vect{F}(\vect{x},t)}{\rho(\vect{x},t)}+\nu \Delta \vect{u}(\vect{x},t)
\end{align}
\end{subequations}

Equation \ref{analytical:navierStokesA}, tell us that the speed is a divergence free field.
Equation \ref{analytical:navierStokesB} is an equation of evolution. The presence of $p$ allow to maintain the speed always divergence free.
Taking the rotational of \ref{analytical:navierStokesB} make disappear the pressure term (because the rotational of a gradient is 0).

The pressure term can be seen as a correction of the divergence without changing the rotational.

We can rewrite Navier-Stokes equation as:
\begin{subequations}
\begin{align}
\label{analytical:navierStokesRewA}
\vect{\nabla} \cdot \vect{v}(\vect{x} ,t)&=0\\
\label{analytical:navierStokesRewB}
\partial_t \vect{v}(\vect{x} ,t)&=f(\vect{v}(\vect{x},t))-\vect{\nabla}\phi
\intertext{Where}
f(\vect{v}(\vect{x},t))&=-(\vect{v}(\vect{x},t)\cdot\vect{\nabla} ) \vect{v}(\vect{x} ,t)+\frac{\vect{F}(\vect{x},t)}{\rho(\vect{x},t)}+\nu \Delta \vect{u}(\vect{x},t)\\
\vect{\nabla}\phi&=\frac{\vect{\nabla}p}{\rho}
\end{align}
\end{subequations}

Taking the divergence of equation \ref{analytical:navierStokesRewB} we found:
\begin{equation}
  \Delta \phi=\vect{\nabla}\cdot f(\vect{v})
\end{equation}

But using the notation of \ref{introduction:projectiondef}:
\begin{equation}
	-\vect{\nabla}\phi=P(f(\vect{v}))
\end{equation}

The equation of Navier-Stokes is then equivalent to solve:
\begin{equation}
  \partial_t \vect{v}(\vect{x} ,t)=f(\vect{v}(\vect{x},t))+P(f(\vect{v}(\vect{x},t))=(\eye+P)f(\vect{v}(\vect{x},t))
\end{equation}

\subsection{Mobile Eulerian}
When topology is variable, we need to add another equation to tell where is the domain of computation.
A natural boundary condition is the free surface, witch move with the speed at given point.

We define $\lambda$ a system of representative. And $\vect{\xi{x}}_{\lambda}$ the position of point given by $\lambda$.
The domain is given by the union of $\vect{\xi}$ for all $\lambda$.

The Navier-Stokes equation are now:
\begin{subequations}
\begin{align}
\label{analytical:navierStokesMobA}
\vect{\nabla} \cdot \vect{v}(\vect{x} ,t)&=0\\
\label{analytical:navierStokesMobB}
\partial_t \vect{v}(\vect{x} ,t)+(\vect{v}(\vect{x},t)\cdot\vect{\nabla} ) \vect{v}(\vect{x} ,t)&=-\frac{\vect{\nabla} p(\vect{x},t)}{\rho(\vect{x},t)}+\frac{\vect{F}(\vect{x},t)}{\rho(\vect{x},t)}+\nu \Delta \vect{u}(\vect{x},t)\\
\partial_t \vect{\xi}_{\lambda}(t)&=\vect{v}(\vect{\xi}_{\lambda},t)
\end{align}
\end{subequations}

We can as in section \ref{analytical:fixe_eulerian} eliminate the constrain on the divergence with the same notation:
\begin{align}
\partial_t \vect{v}(\vect{x} ,t)&=f(\vect{v}(\vect{x},t))+P(f(\vect{v}(\vect{x},t))=(\eye+P)f(\vect{v}(\vect{x},t))\\
\partial_t \vect{\xi}_{\lambda}(t)&=\vect{v}(\vect{\xi}_{\lambda},t)
\end{align}

\subsection{Convective term}
\label{analytical:convectif}
We will in this section rewrite \ref{analytical:navierStokesB} in another form.

We begin to define a new position system $\vect{\xi}_{\lambda}(t)$:
\begin{subequations}
\begin{align}
 \partial_t \vect{\xi}_{\lambda}(t)&=v(\vect{\xi}_{\lambda}(t),t)\\
 \vect{\xi}_{\lambda}(t_0)&=\vect{\xi}^{0}_{\lambda}
\end{align}
\end{subequations}

$\lambda$ is a system of representative that is initialized by initial condition.
For a given $\lambda$, $\vect{\xi}_{\lambda}(t)$ follow the speed at the given point.

This position system is called particle coordinate or Lagrangian.
$\vect{\xi}_\lambda(t)$ is called the characteristic.

We then can define the Lagrangian speed.
\begin{equation}
 \vect{u}_{\lambda}(t)=\vect{v}(\vect{\xi}_{\lambda}(t),t)
\end{equation}

The Lagrangian speed is the speed following a characteristic.

Taking the total derivative(the derivative of everything that depend on time) of the Lagrangian speed with respect of time, we have:
\begin{equation}
\frac{d \vect{u}_{\lambda}(t)}{d t}=\frac{d \vect{v}(\xi_{\lambda},t)}{d t}=\partial_t \vect{v}+\left(\frac{\partial \vect{\xi}_{\lambda}(t)}{\partial t}\cdot\vect{\nabla}\right)\vect{v}
\end{equation}

This can be rewritten:
\begin{equation}
\frac{d \vect{u}_{\lambda}(t)}{d t}=\partial_t \vect{v}+\left(\vect{v} cdot\vect{\nabla}\right)\vect{v}
\end{equation}

This is exactly the lhs of equation \ref{analytical:navierStokesB}.

We then can rewrite Navier-Stokes equation as:
\begin{subequations}
\begin{align}
\label{analytical:navierStokesLagA}
\vect{\nabla} \cdot \vect{u}_{\lambda}(t)&=0\\
\label{analytical:navierStokesLagB}
\frac{d \vect{u}_{\lambda}(t)}{d t}&=-\frac{\vect{\nabla} p(\vect{\xi}_{\lambda}(t),t)}{\rho(\vect{\xi}_{\lambda},t)}+\frac{\vect{F}(\vect{\xi}_{\lambda}(t),t)}{\rho(\vect{x},t)}+\nu \Delta \vect{v}_{\lambda}(t)
\end{align}
\end{subequations}

\begin{rem}
We remark that with this notation some expression like spatial derivative of $\vect{u}$ is not defined.
In an analytical procedure, we can transform $\vect{u}$ to $\vect{v}$ with the help of the expression of $\vect{\xi}$, then the derivative is well defined.
But in a numerical procedure, the point where the value is know are particle in the case of $\vect{u}$. This demand to define derivative on unstructured grid
or to interpolate at grid point.
\end{rem}

