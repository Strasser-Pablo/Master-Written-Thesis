\chapter{Analytical Problem And Analytical Result}

\minitoc
\section*{Introduction}
In this chapter, we will discuss the analytical problem that we want to solve.
The other numerical chapter will consist of numericals schemes to solve the problem given in this chapter.

\section{Variable and Parameter}

The problem will consist to find a function of space and time in a certain domain respecting some condition and equation.

We begin to define space time and domaine:
\begin{description}
\item[The \emph{space}] consist of a vector in fixed Cartesian coordinate with dimension typically 2 or 3 (Navier-Stokes on 1d is trivial).
Space is noted $\vect{x}$.
\item[The \emph{time}] consist of a indice indicating the evolution of the function in time.
Time is noted $t$.
\item[The \emph{domain}] is the spatial region where a field is defined. This consist physically in the spatial region where water lay.
Boundary condition need to be provided.
The domain can vary with respect of time and can vary with respect of the solution at previous time.
The domain don't need to be connected. But can be decomposed in union of connected subdomain.

\begin{example}
 If we take a drop of water falling in a glass of water. The domain consist of the region where the water lie.
 At the beginning this will consist of domain consisting of two connected subdomain (the drop and the glass)
 that after a given time will merge in one domain.
\end{example}

The Domain is generally noted $\Omega$ or $\Omega(t)$ to emphasis that it can be variable with respect of time.
\end{description}

The variable of the equation are named \emph{Field} and consist of a function with respect of space in the spacial domain and time in the temporal domain.
\begin{equation}
 F(x,t)\qquad \text{With $x\in\Omega(t)$ and $t_0\leq t \leq t_f$}
\end{equation}

We have the following field:
\begin{description}
\item[The \emph{speed}:] A vector field consisting of the speed in space and time.
The speed is noted $\vect{v}$.
\item[The \emph{pressure}:] A scalar field consisting of an internal force per unit area in the fluid.
We will not use the pressure in this work, but the pressure divided by density.
If we note the pressure $\tilde{p}$ we use:
\begin{equation}
 p=\frac{\tilde{p}}{\rho}
\end{equation}
Because we use incompressible Navier-Stokes equation, $\rho$ is constant. This can be view as a change of unit.
\end{description}

The following constant parameter are used:
\begin{description}
\item[The \emph{viscosity}] represent how much water particle stick together.
Higher is the viscosity less abrupt speed change are possible.
The viscosity is noted $\mu$. We use allot $\nu$ that is the viscosity divided by the density.
\begin{equation}
 \nu=\frac{\mu}{\rho}
\end{equation}
\item[The \emph{density}] is a mass per unit volume.
Density play the same role as mass in Newton's mechanic.
In the incompressible Navier-Stokes equation density is a constant because incompressibility tell us that density doesn't change.
The density is noted $\rho$.
\item[The \emph{external force}] is a user defined vector field that represent the external force applied to the water.
It can possibly depend on the solution for previous time.
The force is noted $\vect{F}$.
\end{description}

\section{Navier-Stokes equation}

\subsection{Fixed Topology Eulerian}
\label{analytical:fixe_eulerian}
The incompressible Navier-Stokes equation are given by:
\begin{subequations}
\begin{align}
\label{analytical:navierStokesA}
\vect{\nabla} \cdot \vect{v}(\vect{x} ,t)&=0\\
\label{analytical:navierStokesB}
\partial_t \vect{v}(\vect{x} ,t)+(\vect{v}(\vect{x},t)\cdot\vect{\nabla} ) \vect{v}(\vect{x} ,t)&=-\vect{\nabla} p(\vect{x},t)+\frac{\vect{F}(\vect{x},t)}{\rho(\vect{x},t)}+\nu \Delta \vect{u}(\vect{x},t)
\end{align}
\end{subequations}

Equation \ref{analytical:navierStokesA}, tell us that the speed is a divergence free field.
Equation \ref{analytical:navierStokesB} is an equation of evolution. The presence of $p$ allow to maintain the speed always divergence free.
Taking the rotational of \ref{analytical:navierStokesB} make disappear the pressure term (because the rotational of a gradient is 0).

The pressure term can be seen as a correction of the divergence without changing the rotational.

We can rewrite Navier-Stokes equation as:
\begin{subequations}
\begin{align}
\label{analytical:navierStokesRewA}
\vect{\nabla} \cdot \vect{v}(\vect{x} ,t)&=0\\
\label{analytical:navierStokesRewB}
\partial_t \vect{v}(\vect{x} ,t)&=f(\vect{v}(\vect{x},t))-\vect{\nabla}p
\intertext{Where}
f(\vect{v}(\vect{x},t))&=-(\vect{v}(\vect{x},t)\cdot\vect{\nabla} ) \vect{v}(\vect{x} ,t)+\frac{\vect{F}(\vect{x},t)}{\rho(\vect{x},t)}+\nu \Delta \vect{u}(\vect{x},t)
\end{align}
\end{subequations}

Taking the divergence of equation \ref{analytical:navierStokesRewB} we found:
\begin{equation}
  \Delta p=\vect{\nabla}\cdot f(\vect{v})
\end{equation}
But using the notation of \ref{introduction:projectiondef}:
\begin{equation}
	-\vect{\nabla}p=P(f(\vect{v}))
\end{equation}

The equation of Navier-Stokes is then equivalent to solve:
\begin{equation}
  \partial_t \vect{v}(\vect{x} ,t)=f(\vect{v}(\vect{x},t))+P(f(\vect{v}(\vect{x},t))=(\eye+P)f(\vect{v}(\vect{x},t))
\end{equation}
$(\eye+P)$ is a projector on a divergence free space without changing the rotational.

\subsection{Mobile Topology Eulerian}
When topology is variable, we need to add another equation to tell where is the domain of computation.
A natural boundary condition is the free surface, witch move with the speed at given point.

We define $\lambda$ a system of representative. And $\vect{\xi}_{\lambda}$ the position of point given by $\lambda$.
The domain is given by the union of $\vect{\xi}$ for all $\lambda$.

The Navier-Stokes equation are now:
\begin{subequations}
\begin{align}
\label{analytical:navierStokesMobA}
\vect{\nabla} \cdot \vect{v}(\vect{x} ,t)&=0\\
\label{analytical:navierStokesMobB}
\partial_t \vect{v}(\vect{x} ,t)+(\vect{v}(\vect{x},t)\cdot\vect{\nabla} ) \vect{v}(\vect{x} ,t)&=-\vect{\nabla} p(\vect{x},t)+\frac{\vect{F}(\vect{x},t)}{\rho(\vect{x},t)}+\nu \vect{\Delta} \vect{u}(\vect{x},t)\\
\partial_t \vect{\xi}_{\lambda}(t)&=\vect{v}(\vect{\xi}_{\lambda},t)
\end{align}
\end{subequations}
We can as in section \ref{analytical:fixe_eulerian} eliminate the constrain on the divergence with the same notation:
\begin{subequations}
\begin{align}
\partial_t \vect{v}(\vect{x} ,t)&=f(\vect{v}(\vect{x},t))+P(f(\vect{v}(\vect{x},t))=(\eye+P)f(\vect{v}(\vect{x},t))\\
\partial_t \vect{\xi}_{\lambda}(t)&=\vect{v}(\vect{\xi}_{\lambda},t)\label{ana:move:equ}
\end{align}
\end{subequations}

\subsection{Lagrangian form of Navier-Stokes equation}
\label{analytical:convectif}
We will in this section rewrite \ref{analytical:navierStokesB} in another form.
We begin to define a new position system(witch is similar to the particle position used in previous section) $\vect{\xi}_{\lambda}(t)$:
\begin{subequations}
\begin{align}
 \partial_t \vect{\xi}_{\lambda}(t)&=v(\vect{\xi}_{\lambda}(t),t)\\
 \vect{\xi}_{\lambda}(t_0)&=\vect{\xi}^{0}_{\lambda}
\end{align}
\end{subequations}
$\lambda$ is a system of representative that is initialized by initial condition.
For a given $\lambda$, $\vect{\xi}_{\lambda}(t)$ follow the speed at the given point.
This position system is called particle coordinate or Lagrangian.
$\vect{\xi}_\lambda(t)$ is called the characteristic.

We then can define the Lagrangian speed.
\begin{equation}
 \vect{u}_{\lambda}(t)=\vect{v}(\vect{\xi}_{\lambda}(t),t)
\end{equation}
The Lagrangian speed is the speed following a characteristic.

Taking the total derivative(the derivative of everything that depend on time) of the Lagrangian speed with respect of time, we have:
\begin{equation}
\frac{d \vect{u}_{\lambda}(t)}{d t}=\frac{d \vect{v}(\xi_{\lambda},t)}{d t}=\partial_t \vect{v}+\left(\frac{\partial \vect{\xi}_{\lambda}(t)}{\partial t}\cdot\vect{\nabla}\right)\vect{v}
\end{equation}
This can be rewritten:
\begin{equation}
\frac{d \vect{u}_{\lambda}(t)}{d t}=\partial_t \vect{v}+\left(\vect{v} \cdot\vect{\nabla}\right)\vect{v}
\end{equation}
This is exactly the lhs of equation \ref{analytical:navierStokesB}.

We then can rewrite Navier-Stokes equation as:
\begin{subequations}
\begin{align}
\label{analytical:navierStokesLagA}
\vect{\nabla} \cdot \vect{u}_{\lambda}(t)&=0\\
\label{analytical:navierStokesLagB}
\frac{d \vect{u}_{\lambda}(t)}{d t}&=-\vect{\nabla} p(\vect{\xi}_{\lambda}(t),t)+\frac{\vect{F}(\vect{\xi}_{\lambda}(t),t)}{\rho(\vect{x},t)}+\nu \Delta \vect{u}_{\lambda}(t)
\end{align}
\end{subequations}

\begin{remark}
We remark that with this notation some expression like spatial derivative of $\vect{u}$ is not defined.
In an analytical procedure, we can transform $\vect{u}$ to $\vect{v}$ with the help of the expression of $\vect{\xi}$, then the derivative is well defined.
But in a numerical procedure, the point where the value are know are particle. This demand to define derivative on unstructured grid
or to interpolate at grid point.
\end{remark}

\section{Boundary Condition}

We will be interested in two kind of boundary condition.
\begin{description}
 \item[\emph{Inflow} Boundary:]
 An inflow boundary is a source of water.
 \item[\emph{free surface} Boundary:]
 A free surface boundary is a freely moving boundary where the other fluid is inert.
\end{description}

\subsection{Inflow}

Inflow boundary condition have constant prescribed speed.
This give the following constraint on pressure:
\begin{equation}
\nabla p(\vect{x},t)=f(\vect{v}(\vect{x},t))
\end{equation}

\subsection{Free Surface Boundary Condition}
\label{ana:free:surface}
To define the free surface boundary condition, we need to define the stress tensor:
\begin{equation}
	\sigma_{ij}=-p \delta_{ij}+\nu\left(\frac{\partial v_{i}}{\partial x_{j}}+\frac{\partial v_{j}}{\partial x_{i}}\right)
\end{equation}
The boundary condition are then given by(for 3d):
\begin{align}
	\sum_{i,j}\sigma_{ij}n_{i}n_{j}&=0\\
	\sum_{i,j}\sigma_{ij}t^{1}_{i}n_{j}&=0\\
	\sum_{i,j}\sigma_{ij}t^{2}_{i}n_{j}&=0\\
\end{align}
Where $\vect{n}$ are the normal vector and $\vect{t}^1$ and $\vect{t}^2$ are two none-colinear tangent vector.
For 2d only one tangent vector is needed.

The first equation can be used to find boundary condition for $p$.
The other equation only depend of speed, because $\vect{n}$ is orthogonal to $\vect{t}^{1}$ and $\vect{t}^{2}$.
The value of $\nu$ is not relevant for the condition (a global none zero constant).
The constraint is linear.