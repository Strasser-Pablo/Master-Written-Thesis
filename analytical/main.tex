\chapter{Analytical}
\section{Introduction}
In this chapter, i will discuss the analytical problem that we want to solve.
The other numerical chapter will consist of a numerical sheme to solve the problem given in this chapter.

\section{Variable and Parameter}

The problem will consist to find a function of space and time respecting somes condition and equation.

\subsection{Space and Time}

The space consist of a vector in fixed cartesian coordinate with dimension typically 2 or 3 (Navier-Stokes on 1d is trivial).
Space is noted $x$.
The time consist of a indice indicating the evolution of the function in time.
Time is noted $t$.

\subsection{Domain}

A domain is the spatial region where a field is defined. This consist physically in the spatial region where watter lay.
Boundary condition need to be provided.

The Domain can vary with respect of time and can vary with respect of the solution at previous time.

The Domain don't need to be connex. But can be decomposed in union of connex Subdomain.

\begin{example}
 If we take a drop of watter falling in a glass of watter. The domain consist of the region where the watter lie.
 At the begining this will consist of domain consisting of two connex subdomain (the drop and the glass)
 that after a given time will merge in one domain.
\end{example}

The Domain is generaly noted $\Omega$ or $\Omega(t)$ to emphasy that it can be variable with respect of time.

\subsection{Field}

A Field consist of a function with respect space and time of space in the domain and time in the temporal domain.
\begin{equation}
 F(x,t)\qquad \text{With $x\in\Omega(t)$ and $t_0\leq t \leq t_f$}
\end{equation}

\subsubsection{Vector Field}
A vector field is a field that is a vector.

We only have one Vector Field named speed and noted $\vect{v}$.

The speed physically represent the speed of water at a given position.
\subsubsection{Scalar Field}
A scalar field is a field that is a scalar.

We only have on Scalar Field named pressure and noted $p$.

The pressure physically represent an internal force per unit area between water particle.

\subsection{Parameter}

The following parameter are used in this work.

\subsubsection{Viscosity}
The viscosity represent how much water particle stick together.

Higher is the viscosity less abrupt speed change is possible.

The viscosity is noted $\mu$. We use alot $\nu$ that is the viscosity divided by the density.
\begin{equation}
 \nu=\frac{\mu}{\rho}
\end{equation}

\subsubsection{Density}

The density is a mass per unit volumn.

Density play the same role as mass in Newton's mechanic.

In the incompressible Navier-Stokes equation density is more a constant depending
of the domain because incompressibility tell us that density doesn't change.

The density is noted $\rho$.

\subsubsection{Force}

The force is a user defined vector field that represent the external force applyed to the water.
It can possibly depend on the solution for previous time.

The force is noted $\vect{f}$

We often assume that the force is divergence free.
\section{Navier Stokes}

The incompressible Navier-Stokes equation are given by:

\begin{align}
\label{analytical:navierStokesA}
\vect{\nabla} \cdot \vect{v}(\vect{x} ,t)&=0\\
\label{analytical:navierStokesB}
\partial_t \vect{v}(\vect{x} ,t)+(\vect{v}(\vect{x},t)\cdot\vect{\nabla} ) \vect{v}(\vect{x} ,t)&=-\frac{\vect{\nabla} p(\vect{x},t)}{\rho(\vect{x},t)}+\frac{\vect{F}(\vect{x},t)}{\rho(\vect{x},t)}+\nu \Delta \vect{u}(\vect{x},t)
\end{align}

Equation \ref{analytical:navierStokesA}, tell us that the speed is a divergence free field.
Equation \ref{analytical:navierStokesB} is a equation of evolution. The pressence of $p$ allow to maintain the speed always divergence free.
Taking the rotational of \ref{analytical:navierStokesB} make disappear the pressure term (because the rotational of a gradiant is 0).

The pressure term can be seen as a correction of the divergence without changing the rotational.

We can obtain an expression for the pressure by taking the divergence of \ref{analytical:navierStokesB} for $F$ a divergence free force.
\begin{equation}
\label{analytical:pressure}
\Delta p=-\rho \vect{\nabla} \cdot (\vect{v}\vect{\nabla} \cdot \vect{v})
\end{equation}

\subsection{Convectif term}
\label{analytical:convectif}
We will in this section rewrite \ref{analytical:navierStokesB} in nother form.

We begin to define a new position system $\vect{\xi}_{\lambda}(t)$.
\begin{align}
 \partial_t \vect{\xi}_{\lambda}(t)&=v(\xi_{\lambda}(t),t)\\
 \xi_{\lambda}(t_0)&=\xi^{0}_{\lambda}
\end{align}

$\lambda$ is a system of representent that is initialised by initial condition.
For a given $\lambda$ $\xi_{\lambda}(t)$ follow the speed at the given point.

This position system is called particle coordinate or laggragean.
$\xi_\lambda(t)$ is called the characteristic.

We then can define the laggragean speed.
\begin{equation}
 u_{\lambda}(t)=v(\xi_{\lambda}(t),t)
\end{equation}

The laggrangean speed is the speed following a characteristic.

Taking the total derivative(the derivative of everything that depend on time) of the laggrangean speed with respect of time, we have:
\begin{equation}
\frac{d u_{\lambda}(t)}{d t}=\frac{d v(\xi_{\lambda},t)}{d t}=\partial_t \vect{v}+\left(\frac{\partial \vect{\xi_{\lambda}}(t)}{\partial t}\cdot\vect{\nabla}\right)\vect{v}
\end{equation}

This can be rewritten:
\begin{equation}
\frac{d u_{\lambda}(t)}{d t}=\partial_t \vect{v}+\left(\vect{v} cdot\vect{\nabla}\right)\vect{v}
\end{equation}

This is exactly the lhs of equation \ref{analytical:navierStokesB}.

We then can rewrite Navier-Stokes equation as:
\begin{align}
\label{analytical:navierStokesLagA}
\vect{\nabla} \cdot \vect{u}_{\lambda}(t)&=0\\
\label{analytical:navierStokesLagB}
\frac{d u_{\lambda}(t)}{d t}&=-\frac{\vect{\nabla} p(\xi_{\lambda}(t),t)}{\rho(\xi_{\lambda},t)}+\frac{\vect{F}(\xi_{\lambda}(t),t)}{\rho(\vect{x},t)}+\nu \Delta \vect{v}_{\lambda}(t)
\end{align}

\begin{rem}
We remark that with this notation some expression like spatial derivative of $u$ is not defined.
In an analytical procedure, we can transform $u$ to $v$ with the help of the expression of $\xi$, then the derivative is well defined.
But in a numerical procedure, the point where the value is know are particle in the case of $u$. This demand to define derivative on unstructured grid.
Or to interpolate at grid point.
\end{rem}

\subsection{Projection}

We have seen in section \ref{introduction:projection} that every speed can be projected in a divergence free space with:
\begin{align}
 \vect{v_{new}}&=v-\vect{\nabla}p\\
 \Delta p&=\vect{\nabla} \cdot \vect{v}
\end{align}

$p$ defined this way is to a factor $\rho$ the same kind of correction that we find in \ref{analytical:navierStokesB}.
And correct the speed to a divergence free space.

This fact will be used in numerical method.

If $v$ is already divergence free, this will change nothing (with good and consistant boundary condition and discretisation in case of discrete case).
Because the solution of $\Delta p=0$ is a constant pressure.

