\chapter{Fixed}
\minitoc
\section{Introduction}

We will in this chapter be interested in the case of fixed domain Navier-Stokes with grid discretization.
This case is easier than the variable domain case and serve as basis to more complicated case.

\section{Overview}

We will in this chapter present two way to solve Navier-Stokes equation.
We will show that this two method are equivalent if a Runge-Kutta method is used to solve.

The two method are presented at next section.
\subsection{Analytical expression for pressure}
\label{fixed:analytical}

We use the work of section \ref{analytical:fixe_eulerian} that Navier-Stokes equation is analytically equivalent to:
\begin{equation}
  \partial_t \vect{v}(\vect{x} ,t)=(\eye+P)f(\vect{v}(\vect{x},t))
\end{equation}

$\eye+P$ project a none divergence free vector to a divergence free vector.
The solution $\vect{v}$ is a divergence free vector. Because the integral of a divergence free vector is again a divergence free vector.

But a priori the numerical solution is not necessary divergence free.

\subsection{Projection of speed}
\label{fixed:proj}

We can project every speed in a divergence free space:
\begin{subequations}
\begin{align}
  \partial_t \vect{\tilde{v}}(\vect{x} ,t)&=f(\vect{v}(\vect{x},t))\\
  \vect{v}(\vect{x},t)&=(\eye+P)\vect{\tilde{v}}
\end{align}
\end{subequations}

We have the certitude to be always divergence free by construction.

\section{Runge-Kutta}

A Runge-Kutta method is a method to solve the ODE (I drop the $\vect{x}$ dependence, but it's understood that this equation is for a vector formed in concatenating every speed):
\begin{equation}
\partial_t \vect{v}(t)=f(\vect{v})
\end{equation}

The solution at time $n+1$ is given from the solution at time $n$ with $\Delta t$ the time step and $a$, $b$ parameter of the method.
\begin{subequations}
\begin{align}
	\vect{v}_{n+1}&=\vect{v}_{n}+\sum_{i=1}^{s}b_{i}k_{i}\\
	k_{j}&=\Delta t f(\vect{v}_{n}+\sum_{i=1}^{j-1}a_{j,i}k_{i})
\end{align}
\end{subequations}

\section{Runge-Kutta based scheme}
\label{fixed:sect:runge-kutta}
We now integrate method \ref{fixed:analytical} and \ref{fixed:proj} with a Runge-Kutta method.

Scheme for projection at every evaluation of speed:
\begin{subequations}
\begin{align}
\vect{\tilde{v}}_{n+1}&=\vect{\tilde{v}}_{n}+\sum_{i=1}^{s}b_{i}\tilde{k}_{i}+P\left(\vect{\tilde{v}}_{n}+\sum_{i=1}^{s}b_{i}\tilde{k}_{i}\right)\\
\tilde{k}_{i}&=\Delta t f\left(\vect{\tilde{v}}_{n}+\sum_{j=1}^{i-1}a_{ij}\tilde{k}_{j}+P\left(\vect{\tilde{v}}_{n}+\sum_{j=1}^{i-1}a_{ij}\tilde{k}_{j}\right)\right)
\end{align}
\end{subequations}

We project at every estimation of speed. Making by construction every speed divergence free.

Scheme for analytical expression for pressure:
\begin{subequations}
\begin{align}
	\vect{v}_{n+1}&=\vect{v}_{n}+\sum_{i=1}^{s}b_{i}k_{i}\\
	k_{i}&=\Delta t f\left(\vect{v}_{n}+\sum_{j=1}^{i-1}a_{ij}k_{j}\right)+\Delta t P\left(f\left(\vect{v}_{n}+\sum_{j=1}^{i-1}a_{ij}k_{j}\right)\right)
\end{align}
\end{subequations}

The following theorem show that the two scheme are the same.

\begin{theorem}
If $\vect{v}_{n}=\vect{\tilde{v}}_{n}$ and $\vect{v}_n$ is divergence free then $\vect{v}_{n+1}=\vect{\tilde{v}}_{n+1}$ 
\end{theorem}
\begin{proof}
We begin by prove the following lemma.
\begin{lemma}
\begin{equation}
  k_{i}=\tilde{k}_{i}+P(\tilde{k}_{i})
\end{equation}
\end{lemma}
\begin{proof}
By recurrence on $i$.
We begin with $i=1$:
\begin{equation}
  k_{1}=\Delta tf\left(\vect{v}_n\right)+\Delta tP\left(f\left(\vect{v}_n\right)\right)
\end{equation}
\begin{equation}
\tilde{k}_{1}=\Delta tf\left(\vect{v}_n\right)
\end{equation}
\begin{equation}
  k_{1}=\Delta t f\left(\vect{v}_n\right)+\Delta tP\left(f\left(\vect{v}_n\right)\right)=\tilde{k}_1+P\left(\tilde{k}_1\right)
\end{equation}

Assuming true for smaller $i$:
\begin{align*}
  k_{i}&=\Delta tf\left(\vect{v}_n+\sum_{j=1}^{n}a_{ij}k_{j}\right)+\Delta tP\left(f\left(\vect{v}_n+\sum_{j=1}^{i-1}a_{ij}k_{ij}\right)\right)\\
  &=\Delta tf\left(\vect{v}_n+\sum_{j=1}^{i-1}a_{ij}\left(\tilde{k}_{j}+P\left(\tilde{k}_{j}\right)\right)\right)+\Delta tP\left(f\left(\vect{v}_n+\sum_{j=1}^{i-1}a_{ij}\left(\tilde{k}_{j}+P\left(\tilde{k}_{j}\right)\right)\right)\right)\\
  \tilde{k}_{i}&=\Delta tf\left(\vect{v}_n+\sum_{j=1}^{i-1}a_{ij}\tilde{k}_{j}+P\left(\vect{v}_n+\sum_{j=1}^{i-1}a_{ij}\tilde{k}_{j}\right)\right)\\
  &=\Delta tf\left(\vect{v}_n+\sum_{j=1}^{i-1}a_{ij}\left(\tilde{k}_{j}+P\left(\tilde{k}_{j}\right)\right)\right)\\
  k_{i}&=\tilde{k}_{i}+P\left(\tilde{k}_{i}\right)
\end{align*}
\end{proof}

We now write the expression for $\vect{v}_{n+1}$ and $\vect{\tilde{v}}_{n+1}$ using the lemma.

\begin{align*}
\vect{v}_{n+1}&=\vect{v}_{n}+\sum_{i=1}^{s}b_{i}k_{i}\\
&=\vect{v}_{n}+\sum_{i=1}^{s}b_{i}\left(\tilde{k}_{i}+P\left(\tilde{k}_{i}\right)\right)\\
\vect{\tilde{v}}_{n+1}&=\vect{\tilde{v}}_{n}+\sum_{i=1}^{s}b_{i}\tilde{k}_{i}+P\left(\vect{\tilde{v}}_{n}+\sum_{i=1}^{s}b_{i}\tilde{k}_{i}\right)\\
&=\vect{\tilde{v}}_{n}+\sum_{i=1}^{s}b_{i}\left(\tilde{k}_{i}+P\left(\tilde{k}_{i}\right)\right)\\
\intertext{Using that $\vect{v}_{n}=\tilde{\vect{v}_{n}}$}
\vect{v}_{n+1}&=\vect{\tilde{v}}_{n+1}
\end{align*}

\end{proof}

\begin{theorem}
For the implicit Runge-Kutta case, where the sum to $i-i$ go now to $s$.
If $\vect{v}_{n}=\vect{\tilde{v}}_{n}$ and $\vect{v}_n$ is divergence free then $\vect{v}_{n+1}=\vect{\tilde{v}}_{n+1}$ 
\end{theorem}
\begin{proof}
We begin by prove the following lemma.
\begin{lemma}
After relabeling noting $k_{i}^{j}$ the $j$ solution of $k_{i}$.
We have for every $j$.
\begin{equation}
  k_{i}=\tilde{k}_{i}^{j}+P(\tilde{k}_{i}^{j})
\end{equation}
This doesn't say that we have existence and unicity of the solution only that if we have a set of solution for $k$ we have a bijection
to solution of $\tilde{k}$
\end{lemma}
\begin{proof}
Because the sum goes to $s$ and not $i-1$ we cannot use recurrence.
We now will use vector notation for $k$.

We construct $k$ and $\tilde{k}$ vector with the vector $k_{i}$ and $\tilde{k}_{i}$
\begin{align}
k&=\begin{pmatrix}
    k_{1}\\
    \vdots\\
    k_{s}\\
  \end{pmatrix}\\
\tilde{k}&=\begin{pmatrix}
    \tilde{k}_{1}\\
    \vdots\\
    \tilde{k}_{s}\\
  \end{pmatrix}
\end{align}

We form $\hat{P}$ and $\hat{f}$ from a Kronecker product:
\begin{align}
\hat{P}&=\eye_{s}\kron P=\begin{pmatrix}P	&\ldots	&0\\
			\vdots &\ddots 	&\vdots\\
			0	&0	&P\\
	\end{pmatrix}\\
\hat{f}&=\eye_{s}\kron f\begin{pmatrix}f	&\ldots	&0\\
			\vdots &\ddots 	&\vdots\\
			0	&0	&f\\
	\end{pmatrix}
\end{align}

We now define $\hat{v}_n$:
\begin{equation}
\hat{v}_{n}=\begin{pmatrix}
	      v_{n}\\
	      \vdots\\
	      v_{n}
	      \end{pmatrix}
\end{equation}

We now define $\hat{A}$ with the following Kronecker product:
\begin{equation}
\hat{A}=\begin{pmatrix}
    a_{11}	&\ldots	&a_{1s}\\
    \vdots	&\ddots	&\vdots\\
    a_{s1}	&\ldots	&a_{ss}\\
  \end{pmatrix} \kron \eye=A \kron \eye
\end{equation}

$\hat{A}$ and $\hat{P}$ are commutative, because of the mixed-product property.
\begin{align}
\hat{A}\hat{P}&=\left(A\kron \eye \right)
  \left(\eye_{s}\kron P\right)=
	A\kron P\\
    %
    \hat{P}\hat{A}&=
  \left(\eye_{s}\kron P\right)
	\left(A \kron \eye \right)=
	A\kron P
\end{align}

\begin{align}
k&=(1+\hat{P})\Delta t\hat{f}(\hat{v}_{n}+\hat{A}k)\\
\tilde{k}&=\Delta t \hat{f}((1+\hat{P})(\hat{v}_{n}+\hat{A}k))
\end{align}

We begin to define 2 functions:
\begin{align}
k&=g(k)\\
\tilde{k}&=\tilde{g}(\tilde{k})\\
g(x)&=(1+\hat{P})\Delta t \hat{f}(\hat{v}_{n}+\hat{A}x)\\
\tilde{g}(x)&=\Delta t \hat{f}((1+\hat{P})(\hat{v}_{n}+\hat{A}x))
\end{align}

$k$ and $\tilde{k}$ are fixed point of function $g$ and $\tilde{g}$.

We now calculate:
\begin{align}
(1+\hat{P})\tilde{g}(x)&=(1+\hat{P})\Delta t \hat{f}((1+\hat{P})(\hat{v}_{n}+\hat{A}x))=(1+\hat{P})\Delta t \hat{f}(\hat{v}_{n}+\hat{A} (1+\hat{P})x)\\
&=g((1+\hat{P})x)
\end{align}

We now substitute $g$ in $\tilde{g}$
\begin{equation}
(1+\hat{P})\tilde{k}=g((1+\hat{P})\tilde{k})
\end{equation}

$(1+\hat{P})\tilde{k}$ is a fixed point of $g$.
But fixed point of $g$ is $k$.

This directly give the result with $k^{j}$ the $j$ solution:
\begin{equation}
k^{j}=(1+\hat{P})\tilde{k}^{j}
\end{equation}
This is exactly the vector version of what we wanted to prove.
\end{proof}

The rest is exactly the same than in the first case.

To not overload the proof with notation, we drop the $j$ label to indicate witch solution we take.
We now write the expression for $\vect{v}_{n+1}$ and $\vect{\tilde{v}}_{n+1}$ using the lemma.
\begin{align*}
\vect{v}_{n+1}&=\vect{v}_{n}+\sum_{i=1}^{s}b_{i}k_{i}\\
&=\vect{v}_{n}+\sum_{i=1}^{s}b_{i}\left(\tilde{k}_{i}+P\left(\tilde{k}_{i}\right)\right)\\
\vect{\tilde{v}}_{n+1}&=\vect{\tilde{v}}_{n}+\sum_{i=1}^{s}b_{i}\tilde{k}_{i}+P\left(\vect{\tilde{v}}_{n}+\sum_{i=1}^{s}b_{i}\tilde{k}_{i}\right)\\
&=\vect{\tilde{v}}_{n}+\sum_{i=1}^{s}b_{i}\left(\tilde{k}_{i}+P\left(\tilde{k}_{i}\right)\right)\\
\intertext{Using that $\vect{v}_{n}=\tilde{\vect{v}_{n}}$}
\vect{v}_{n+1}&=\vect{\tilde{v}}_{n+1}
\end{align*}

\end{proof}

\begin{corollary}
Solving Navier-Stokes equation with the analytical good choice of pressure.
Is exactly divergence free at every approximation of speed.
\end{corollary}
\begin{proof}
  The expression of $v$ is the solution of the Navier-Stokes equation with the good choice of pressure.
  $\tilde{v}$ is the solution of Runge-Kutta equation without pressure witch is projected at every speed estimation.
  This is divergence free.
  $v$ and $\tilde{v}$ are the same. With end the proof.
\end{proof}

\begin{corollary}
  Using the expression for $v$ or $\tilde{v}$ we are $k$ order precise in time. Where $k$ is the order of the Runge-Kutta method.
\end{corollary}
\begin{proof}
  $v$ and $\tilde{v}$ are the same.
  $v$ consist only of a Runge-Kutta method with a good chosen function.
  So the order in time of $v$ and $\tilde{v}$ is the same that the method order.
\end{proof}


\section{Runge-Kutta based scheme Affine Projection}
\label{fixed:sect:runge-kutta:affine}
We will repeat the work in section \ref{fixed:sect:runge-kutta} but where the projection is an affine operator.
We use the notation of \ref{introduction:projectiondef}.

Scheme for projection at every evaluation of speed:
\begin{subequations}
\begin{align}
\vect{\tilde{v}}_{n+1}&=\vect{\tilde{v}}_{n}+\sum_{i=1}^{s}b_{i}\tilde{k}_{i}+P\left(\vect{\tilde{v}}_{n}+\sum_{i=1}^{s}b_{i}\tilde{k}_{i},\tilde{T}_0\right)\\
\tilde{k}_{i}&=\Delta t f\left(\vect{\tilde{v}}_{n}+\sum_{j=1}^{i-1}a_{ij}\tilde{k}_{j}+P\left(\vect{\tilde{v}}_{n}+\sum_{j=1}^{i-1}a_{ij}\tilde{k}_{j},\tilde{T}_i\right)\right)
\end{align}
\end{subequations}

We project at every estimation of speed. Making by construction every speed divergence free.

Scheme for analytical expression for pressure:
\begin{subequations}
\begin{align}
	\vect{v}_{n+1}&=\vect{v}_{n}+\sum_{i=1}^{s}b_{i}k_{i}\\
	k_{i}&=\Delta t f\left(\vect{v}_{n}+\sum_{j=1}^{i-1}a_{ij}k_{j}\right)+\Delta t P\left(f\left(\vect{v}_{n}+\sum_{j=1}^{i-1}a_{ij}k_{j}\right),T_{i}\right)
\end{align}
\end{subequations}

The following theorem show that the two scheme are the same. And given the relation between $T$ and $\tilde{T}$.

\begin{theorem}
If $\vect{v}_{n}=\vect{\tilde{v}}_{n}$ and $\vect{v}_n$ is divergence free then $\vect{v}_{n+1}=\vect{\tilde{v}}_{n+1}$.
With the relation for the translation:
\begin{align}
\intertext{For $i\geq s$}
  \tilde{T}_i&=\sum_{j=1}^{i-1}a_{ij}\Delta t T_{j}\\
  \tilde{T}_0&=\sum_{i=1}^{s}b_{i}\Delta t T_{i}
\end{align}

\end{theorem}
\begin{proof}
We begin by prove the following lemma.
\begin{lemma}
\begin{equation}
  k_{i}=\tilde{k}_{i}+P(\tilde{k}_{i},\Delta t T_{i})
\end{equation}
\end{lemma}
\begin{proof}
By recurrence on $i$.
We begin with $i=1$:
\begin{equation}
  k_{1}=\Delta tf\left(\vect{v}_n\right)+\Delta tP\left(f\left(\vect{v}_n\right),T_{1}\right)
\end{equation}
\begin{equation}
\tilde{k}_{1}=\Delta tf\left(\vect{v}_n\right)
\end{equation}
\begin{equation}
  k_{1}=\Delta t f\left(\vect{v}_n\right)+\Delta tP\left(f\left(\vect{v}_n\right),T_{1}\right)=\tilde{k}_1+P\left(\tilde{k}_1,\Delta t T_{1}\right)
\end{equation}

Assuming true for smaller $i$:
\begin{align*}
  k_{i}&=\Delta tf\left(\vect{v}_n+\sum_{j=1}^{n}a_{ij}k_{j}\right)+\Delta tP\left(f\left(\vect{v}_n+\sum_{j=1}^{i-1}a_{ij}k_{j}\right),T_{i}\right)\\
  &=\Delta tf\left(\vect{v}_n+\sum_{j=1}^{i-1}a_{ij}\left(\tilde{k}_{j}+P\left(\tilde{k}_{j},\Delta t T_{j}\right)\right)\right)+\Delta tP\left(f\left(\vect{v}_n+\sum_{j=1}^{i-1}a_{ij}\left(\tilde{k}_{j}+P\left(\tilde{k}_{j},\Delta t T_{j}\right)\right)\right),T_{i}\right)\\
  \tilde{k}_{i}&=\Delta tf\left(\vect{v}_n+\sum_{j=1}^{i-1}a_{ij}\tilde{k}_{j}+P\left(\vect{v}_n+\sum_{j=1}^{i-1}a_{ij}\tilde{k}_{j},\tilde{T}_{i}\right)\right)\\
  &=\Delta tf\left(\vect{v}_n+\sum_{j=1}^{i-1}a_{ij}\left(\tilde{k}_{j}+P\left(\tilde{k}_{j},\hat{T}_{j}^{i}\right)\right)\right)\\
  \tilde{T_i}&=\sum_{j}^{i-1}a_{ij}\hat{T}_{j}^{i}\\
  \hat{T}^{i}_{j}&=\Delta t T_{j}\\
  k_{i}&=\tilde{k}_{i}+P\left(\tilde{k}_{i},\Delta tT_j\right)
\end{align*}
This prove the relation and give the relation between the two.
\begin{equation}
  \tilde{T_i}=\sum_{j}^{i-1}a_{ij}\Delta t T_{j}
\end{equation}
\end{proof}

We now write the expression for $\vect{v}_{n+1}$ and $\vect{\tilde{v}}_{n+1}$ using the lemma.

\begin{align*}
\vect{v}_{n+1}&=\vect{v}_{n}+\sum_{i=1}^{s}b_{i}k_{i}\\
&=\vect{v}_{n}+\sum_{i=1}^{s}b_{i}\left(\tilde{k}_{i}+P\left(\tilde{k}_{i},\Delta tT_{i}\right)\right)\\
\vect{\tilde{v}}_{n+1}&=\vect{\tilde{v}}_{n}+\sum_{i=1}^{s}b_{i}\tilde{k}_{i}+P\left(\vect{\tilde{v}}_{n}+\sum_{i=1}^{s}b_{i}\tilde{k}_{i},\tilde{T}_{0}\right)\\
&=\vect{\tilde{v}}_{n}+\sum_{i=1}^{s}b_{i}\left(\tilde{k}_{i}+P\left(\tilde{k}_{i},\hat{T}_{i}^{0}\right)\right)\\
\tilde{T}_{0}&=\sum_{i=1}^{s}b_{i}\hat{T}_{i}^{0}\\
\tilde{T}_{0}&=\sum_{i=1}^{s}b_{i}\Delta t T_{i}\\
\intertext{Using that $\vect{v}_{n}=\tilde{\vect{v}_{n}}$}
\vect{v}_{n+1}&=\vect{\tilde{v}}_{n+1}
\end{align*}

This end the proof.

\end{proof}

\begin{theorem}
For the implicit Runge-Kutta case, where the sum on $i-1$ go now to $s$.
If $\vect{v}_{n}=\vect{\tilde{v}}_{n}$ and $\vect{v}_n$ is divergence free then $\vect{v}_{n+1}=\vect{\tilde{v}}_{n+1}$.
And the translation is given by:
\begin{align}
  \tilde{T_{i}}&=\Delta t\sum_{j}a_{ij}T_{j}\\
  \tilde{T_{0}}&=\Delta t\sum_{j}b_{j}T_{j}
\end{align}

\end{theorem}
\begin{proof}
We begin by prove the following lemma.
\begin{lemma}
After relabeling noting $k_{i}^{j}$ the $j$ solution of $k_{i}$.
We have for every $j$.
\begin{equation}
  k_{i}=\tilde{k}_{i}^{j}+P(\tilde{k}_{i}^{j},\Delta t T_{i})
\end{equation}
This doesn't say that we have existence and unicity of the solution only that if we have a set of solution for $k$ we have a bijection
to solution of $\tilde{k}$
\end{lemma}
\begin{proof}
Because the sum goes to $s$ and not $i-1$ we cannot use recurrence.
We now will use vector notation for $k$.

We construct $k$ and $\tilde{k}$ vector with the vector $k_{i}$ and $\tilde{k}_{i}$
\begin{align}
k&=\begin{pmatrix}
    k_{1}\\
    \vdots\\
    k_{s}\\
  \end{pmatrix}\\
\tilde{k}&=\begin{pmatrix}
    \tilde{k}_{1}\\
    \vdots\\
    \tilde{k}_{s}\\
  \end{pmatrix}
\end{align}

We form $\hat{L}$ the linear part of the projection from $L$ the linear part from the projection $P$.
\begin{align}
\hat{L}&=\eye_{s}\kron L=\begin{pmatrix}L	&\ldots	&0\\
			\vdots &\ddots 	&\vdots\\
			0	&0	&L\\
	\end{pmatrix}
\end{align}


$\tau_{i}$ the translation vector from $T_i$ the translation part of $P$.
$\tau$ is the sum of all individual translation.
\begin{align}
	\tau_i&=\begin{pmatrix}
			0\\
			\vdots\\
			T_{i}\\
			\vdots\\
			0\\
	\end{pmatrix}
	\\
	\tau&=\sum_{i}\tau_{i}=\begin{pmatrix}
			T_{1}\\
			\vdots\\
			T_{s}\\
	\end{pmatrix}
\end{align}

Same for $\tilde{\tau}_{i}$ with $T_{i}$ changed with $\tilde{T}_i$

We now form the function $\hat{f}$ from $f$ with:
\begin{align}
\hat{f}&=\eye_{s}\kron f=\begin{pmatrix}f	&\ldots	&0\\
			\vdots &\ddots 	&\vdots\\
			0	&0	&f\\
	\end{pmatrix}
\end{align}

We now define $\hat{v}_n$:
\begin{equation}
\hat{v}_{n}=\begin{pmatrix}
	      v_{n}\\
	      \vdots\\
	      v_{n}
	      \end{pmatrix}
\end{equation}

We now define $\hat{A}$ with the following Kronecker product:
\begin{equation}
\hat{A}=\begin{pmatrix}
    a_{11}	&\ldots	&a_{1s}\\
    \vdots	&\ddots	&\vdots\\
    a_{s1}	&\ldots	&a_{ss}\\
  \end{pmatrix} \kron \eye=A \kron \eye
\end{equation}

$\hat{A}$ and $\hat{L}$ are commutative, because of the mixed-product property.
\begin{align}
\hat{A}\hat{L}&=\left(A\kron \eye \right)
  \left(\eye_{s}\kron L\right)=
	A\kron L\\
    %
    \hat{L}\hat{A}&=
  \left(\eye_{s}\kron L\right)
	\left(A \kron \eye \right)=
	A\kron L
\end{align}

\begin{align}
k&=(1+\hat{L})\Delta t\hat{f}(\hat{v}_{n}+\hat{A}k)+\Delta t\tau\\
\tilde{k}&=\Delta t \hat{f}((1+\hat{L})(\hat{v}_{n}+\hat{A}k)+\tilde{\tau})
\end{align}

We begin to define 2 functions:
\begin{align}
k&=g(k)\\
\tilde{k}&=\tilde{g}(\tilde{k})\\
g(x)&=(1+\hat{L})\Delta t \hat{f}(\hat{v}_{n}+\hat{A}x)+\Delta t\tau\\
\tilde{g}(x)&=\Delta t \hat{f}((1+\hat{L})(\hat{v}_{n}+\hat{A}x)+\tilde{\tau})
\end{align}

$k$ and $\tilde{k}$ are fixed point of function $g$ and $\tilde{g}$.

We now calculate:
\begin{align}
(1+\hat{L})\tilde{g}(x)&=(1+\hat{L})\Delta t \hat{f}((1+\hat{L})(\hat{v}_{n}+\hat{A}x)+\tilde{\tau})=(1+\hat{L})\Delta t \hat{f}(\hat{v}_{n}+\hat{A} (1+\hat{L})x+\tilde{\tau})\\
&=(1+\hat{L})\Delta t \hat{f}(\hat{v}_{n}+\hat{A} ((1+\hat{L})x+\hat{A}^{-1}\tilde{\tau}))=g((1+\hat{L})x+\tilde{A}^{-1}\tilde{\tau})-\Delta t \tau
\end{align}

We now substitute $g$ in $\tilde{g}$
\begin{equation}
(1+\hat{L})\tilde{k}+\Delta t \tau=g((1+\hat{L})\tilde{k}+\hat{A}^{-1}\tilde{\tau})
\end{equation}

If
\begin{equation}
  \tilde{\tau}=\Delta t \hat{A}\tau
\end{equation}

$(1+\hat{L})\tilde{k}+\Delta t \tau$ is a fixed point of $g$.
But fixed point of $g$ is $k$.

This directly give the result with $k^{j}$ the $j$ solution:
\begin{equation}
k^{j}=(1+\hat{L})\tilde{k}^{j}+\Delta t \tau
\end{equation}
This is exactly the vector version of what we wanted to prove.
\end{proof}

The rest is exactly the same than in the first case.

To not overload the proof with notation, we drop the $j$ label to indicate witch solution we take.
We now write the expression for $\vect{v}_{n+1}$ and $\vect{\tilde{v}}_{n+1}$ using the lemma.
\begin{align*}
\vect{v}_{n+1}&=\vect{v}_{n}+\sum_{i=1}^{s}b_{i}k_{i}\\
&=\vect{v}_{n}+\sum_{i=1}^{s}b_{i}\left(\left(\eye+L\right)\tilde{k}_i+\Delta t T_{i}\right)\\
\vect{\tilde{v}}_{n+1}&=\vect{\tilde{v}}_{n}+\sum_{i=1}^{s}b_{i}\tilde{k}_{i}+L\left(\vect{\tilde{v}}_{n}+\sum_{i=1}^{s}b_{i}\tilde{k}_{i}\right)+\tilde{T}_0\\
&=\vect{\tilde{v}}_{n}+\sum_{i=1}^{s}b_{i}\left(\eye+L\right)\tilde{k}_{i}+\tilde{T}_0\\
\tilde{T}_0&=\Delta t\sum_{i=1}^{s}b_{i}T_{i}\\
\intertext{Using that $\vect{v}_{n}=\tilde{\vect{v}_{n}}$}
\vect{v}_{n+1}&=\vect{\tilde{v}}_{n+1}
\end{align*}

\end{proof}

\section{Spatial discretization}

\subsection{Non-Staggered grid grid}

On an Non-Staggered grid grid we put the variable like figure \ref{fixed:unstaggered}.

\begin{figure}
\directlua{dofile('fixed/unstaggered.lua')}
\caption{Position of variable for Non-Staggered grid grid}
\label{fixed:unstaggered}
\end{figure}

The problem with Non-Staggered grid grid, is that obvious center difference discretization lead to wrong divergence free vector.

We define the derivative of a scalar by:
\begin{equation}
  \partial_x a_i=\frac{a_{i+1}-a{i-1}}{\Delta x}
\end{equation}

The divergence discretization is then given by (notation for 2d, but for 3d it's the same):
\begin{equation}
  \vect{\nabla} \cdot \vect{v}_{i,j}=\frac{{v_{x}}_{i+1,j}-{v_{x}}_{i-1,j}}{\Delta x}+\frac{{v_{y}}_{i+1,j}-{v_{y}}_{i-1,j}}{\Delta y}
\end{equation}

The problem can be see in figure \ref{fixed:unstaggered_div} is that the divergence at red point only use value at blue point and not the point at center (red point).
This allow to have none physical solution that are divergence free.

\begin{figure}
\directlua{dofile('fixed/unstaggered_div.lua')}
\caption{Value used to calculate the divergence at red point are in blue}
\label{fixed:unstaggered_div}
\end{figure}

In figure \ref{fixed:unstaggered_div2} we have solution that is numerically divergence free. But this solution is not continuous and we expect that is not divergence free.

\begin{figure}
\directlua{dofile('fixed/unstaggered_div2.lua')}
\caption{Speed that are numerically divergence free but shouldn't}
\label{fixed:unstaggered_div2}
\end{figure}

The problem with this kind of solution, is that it's very stable.
The projection operator $P$ will be 0, and don't correct this kind of speed. We cannot correct it in general for every choice of $f$.

We can recall one of the formulation of the so call Murphy's law, ``If anything can go wrong, it will.''

We can correct this problem with using a forward or backward expression. But this kind of expression is only first order accurate.

Another solution is to use another grid.

\subsection{Staggered grid}

On a staggered grid we put the variable in the position shown if figure \ref{fixed:staggered}.

\begin{figure}
\directlua{dofile('fixed/staggered.lua')}
\caption{Position of variable on a staggered grid}
\label{fixed:staggered}
\end{figure}

This can seem to be strange to have different component of speed at different place. But it make discretization more easy.

To label variable, we label by cell and then took the lower,left component (cf. figure \ref{fixed:staggered_label}).

\begin{figure}
\directlua{dofile('fixed/staggered_label.lua')}
\caption{Labeling on staggered grid}
\label{fixed:staggered_label}
\end{figure}

We now define the divergence and the gradient:
\begin{align}
  \vect{\nabla}p_{ij}&=\begin{pmatrix}
    \frac{p_{i,j}-p_{i-1,j}}{\Delta x}\\
    \frac{p_{i,j}-p_{i,j-1}}{\Delta y}\\
                      \end{pmatrix}\\
  \vect{\nabla}\cdot \vect{v}_{i,j}&=\frac{{v_{x}}_{i+1,j}-{v_{x}}_{i,j}}{\Delta x}+\frac{{v_{y}}_{i,j+1}-{v_{y}}_{i,j}}{\Delta y}
\end{align}

This two expression are geometrically very clear.

In figure \ref{fixed:staggered_gradient}, the gradient at red speed component use component at blue point.
Contrary at case with center difference we cannot create a none constant vector with 0 gradient.

\begin{figure}
\directlua{dofile('fixed/staggered_gradient.lua')}
\caption{Point used to calculate the red gradient component are in blue}
\label{fixed:staggered_gradient}
\end{figure}

The divergence is given by the sum of inflow around a cell.
In figure \ref{fixed:staggered_divergence}, the divergence at red point use component at blue point.

\begin{figure}
\directlua{dofile('fixed/staggered_divergence.lua')}
\caption{Point used to calculate red divergence are in blue}
\label{fixed:staggered_divergence}
\end{figure}

We cannot have the same situation than without staggered grid, where an nonphysical vector is divergence free.

\subsubsection{Discretization}

We now look at different possible scheme for staggered grid discretization.

\paragraph{Gradient}
\label{fixed:gradient}
The gradient is discretized with center difference.

\begin{equation}
  \vect{\nabla}p_{i,j}=\begin{pmatrix}
                       	\frac{p_{i-1,j}-p_{i,j}}{\Delta x}\\
                       	\frac{p_{i,j-1}-p_{i,j}}{\Delta y}\\
                       \end{pmatrix}
\end{equation}

Center difference is second order accurate to the contrary to forward or backward difference.

\paragraph{Divergence}
\label{fixed:divergence}
The divergence is discretized with center difference.

\begin{equation}
	\vect{\nabla}\cdot \vect{v}_{i,j}=\frac{{v_{x}}_{i,j}-{v_{x}}_{i-1,j}}{\Delta x}+\frac{{v_{y}}_{i,j}-{v_{y}}_{i,j-1}}{\Delta y}
\end{equation}

Because of center difference, it's second order accurate.

\paragraph{Laplacian}
\label{fixed:Laplacian}
The discretization of the Laplacian is the combination of the application of the gradient and divergence.

\begin{equation}
  \Delta p_{i,j}=\frac{p_{i+1,j}-2p_{i,j}+p_{i-1,j}}{(\Delta x)^2}+\frac{p_{i,j+1}-2p_{i,j}+p_{i,j-1}}{(\Delta y)^2}
\end{equation}

This formula is of order 2.

We prove it for the one dimensional case:
\begin{align*}
  f(x+\Delta x)-2f(x)+f(x-\Delta x)&=\\
  \intertext{We make a Taylor expansion around $x$}
  &=f(x)+\Delta x f'(x) +(\Delta x)^2 \frac{f''(x)}{2}+(\Delta x)^3 \frac{f'''(x)}{6}+O((\Delta x)^4)-2f(x)\\
  &\qquad+f(x)-\Delta x f'(x) +(\Delta x)^2 \frac{f''(x)}{2}-(\Delta x)^3 \frac{f'''(x)}{6}\\
  &=(\Delta x)^2f''(x) +O((\Delta x)^4)\\
  f''(x)&=\frac{f(x+\Delta x)-2f(x)+f(x-\Delta x)}{(\Delta x)^2}+O((\Delta x)^2)
\end{align*}

\paragraph{Vector Laplacian}
\label{fixed:vect:laplacian}
For vector Laplacian we use the same scheme as in section \ref{fixed:Laplacian} but for every component of speed.

Staggered grid don't pose problem because we don't have mixed term with derivative and none derivative.

\paragraph{Convection}

Convection term is the more difficult term to treat, because it's none linear and use mixed term.

We need to consider two type of term. Let $i\neq j$. And $\vect{k}$ be a integer vector of the position on the grid.
And $\vect{e}_i$ the i\th unit vector.

We first consider as i\th component:
\begin{equation}
\left(v_{i}\partial_{i}v_{i}\right)^{i}_{\vect{k}}
\end{equation}

We can discretize this in two manner:

\subparagraph{Central difference}
\begin{equation}
{v_{i}}_{\vect{k}}\frac{{v_{i}}_{\vect{k}+\vect{e}_i}-{v_{i}}_{\vect{k}-\vect{e}_i}}{2\Delta x_{i}}
\end{equation}

\subparagraph{Upwind difference}
\label{fixed:upwind}

We use the same method, but instead to use center difference, we use upwind:

\begin{align}
\intertext{if ${v_{i}}_{\vect{k}}<0$}
\left(v_{i}\partial_{i}v_{i}\right)_{\vect{k}}&={v_{i}}_{\vect{k}}\frac{{v_{i}}_{\vect{k}+\vect{e}_i}-{v_{i}}_{\vect{k}}}{\Delta x_{i}}\\
\intertext{if ${v_{i}}_{\vect{k}}>0$}
\left(v_{i}\partial_{i}v_{i}\right)_{\vect{k}}&={v_{i}}_{\vect{k}}\frac{{v_{i}}_{\vect{k}}-{v_{i}}_{\vect{k}-\vect{e}_i}}{\Delta x_{i}}\\
\intertext{if ${v_{i}}_{\vect{k}}=0$}
\left(v_{i}\partial_{i}v_{i}\right)_{\vect{k}}&=0
\end{align}

Upwind is considered more stable (can be proof easily in one dimension for burger's equation) but more diffusive.

We now consider:
\begin{equation}
\left(v_{j}\partial_{j}v_{i}\right)^{i}_{\vect{k}}
\end{equation}

\begin{remark}
  Contrary to the other case, we need to have the speed at a position not know for $v_{j}$. This come because of the staggered grid sheme.
  For this, we can interpolated the know speed of neighbor cell taking for example the average of the 4 neighbor (cf. figure \ref{fixed:staggered_convection_upwind}).
  We will use as notation $v_{j}^{i}$ for the speed component $j$ at position for the component $i$.
  
  \begin{figure}
    \directlua{dofile('fixed/staggered_convection_upwind.lua')}
    \caption{We need the value of the red arrow. For this we can interpolated from the blue arrow.}
    \label{fixed:staggered_convection_upwind}
    \end{figure}
\end{remark}

We can discretize this in two manner:

\subparagraph{Central difference}
\begin{equation}
\left(v_{j}\partial_{j}v_{i}\right)^{i}_{\vect{k}}={v_{j}^{i}}_{\vect{k}}\frac{{v_{i}}_{\vect{k}+\vect{e}_j}-{v_{i}}_{\vect{k}-\vect{e}_j}}{2\Delta x_{j}}
\end{equation}

\subparagraph{Upwind difference}

We use the same method, but instead to use center difference, we use upwind:

\begin{align}
\intertext{if ${v_{j}^{i}}_{\vect{k}}<0$}
\left(v_{j}\partial_{j}v_{i}\right)^{i}_{\vect{k}}&={v_{j}^{i}}_{\vect{k}}\frac{{v_{i}}_{\vect{k}+\vect{e}_j}-{v_{i}}_{\vect{k}}}{\Delta x_{j}}\\
\intertext{if ${v_{j}^{i}}_{\vect{k}}>0$}
\left(v_{j}\partial_{j}v_{i}\right)^{i}_{\vect{k}}&={v_{j}^{i}}_{\vect{k}}\frac{{v_{i}}_{\vect{k}}-{v_{i}}_{\vect{k}-\vect{e}_j}}{\Delta x_{j}}\\
\intertext{if ${v_{j}^{i}}_{\vect{k}}=0$}
\left(v_{j}\partial_{j}v_{i}\right)^{i}_{\vect{k}}&=0
\end{align}

Upwind is considered more stable (can be proven easily in one dimension for burger's equation) but more diffusive.

\section{Pseudo Code}

The pseudo code is not very complicate in principle.
We need to put all mathematical discretization and piece together.

We will use the following order of operation:
\begin{enumerate}
\item Calculate optimal time step.
 \item Use a Runge-Kutta algorithm. We now have two choice:
 \begin{enumerate}
  \item We project the acceleration.
	For this we do the following for every Runge-Kutta step:
	\begin{enumerate}
	\item We set the speed to the wanted boundary condition.
	 \item We calculate the acceleration from discretization of the function noted $f(\vect{v})$ in this section.
	 \item We project the given acceleration.
	\end{enumerate}
 \item We project the speed.
 For this we apply the Runge-Kutta step as in section \ref{fixed:sect:runge-kutta}:
	\begin{enumerate}
	\item We set the speed to the wanted boundary condition.
	\item We obtain a speed estimation from $f(\vect{v})$.
	\item We project the speed estimation to be divergence free.
	\end{enumerate}
 \end{enumerate}

\end{enumerate}

The principal difference of the two method are:
\begin{itemize}
 \item If we project on speed we are always divergence free.
 \item If we project on force, we need to have initial speed that are divergence free. Numerical error can accumulate.
 We can eliminate them with a projection of speed.
 \item Boundary condition need to be transformed as in section \ref{fixed:sect:runge-kutta:affine}.
\end{itemize}

We will in the next section show the implementation of principal part.

\subsection{Pseudo code semantic and notation.}

The notation of pseudo code is inspired of \textbf{C-like} language.

Variable are introduced when used by giving there name.
Affectation are done with $\gets$ witch indicate that the left variable take the right value.

Boolean operation are done with there standard mathematical notation ($=$, $\neq$,$<$,$>$,$\leq$,$\geq$).

Arithmetic operation use the standard mathematical operation ($+$,$-$,$\cdot$,$\frac{x}{x}$).

The following special control flow operation are used:
\begin{description}
 \item[$\mathop{\textbf{Loop}}$:] An infinite loop, is only stopped by a $\Break$ or \Return statement.
 \item[$\mathop{\textbf{For all}}$:] A loop witch take all element of a container. $\mathop{\textbf{for all}}$ expression are normally good candidate for parallelism.
\end{description}

\subsubsection{Container}

The following container will be used:
\begin{enumerate}
 \item[Grid:] Consist of all cell. Used in expression like $\mathop{\textbf{for all}}(it \in Grid)$.
 \item[Unordered Stack:] A container where element can be added and accessed in a $\mathop{\textbf{for all}}$. The order are not important.
 And after usage, the stack is no more used.
 A new Unordered Stack is created with the function call $\FuncCall{newunorderedstack}()$.
 \item[Set:] A set is a container witch is like a mathematical set, where element can be added and each element accessed.
 The difference with a Stack is that in a set duplicate element are stored once.
 A new set is created with the function call $\FuncCall{newset}()$.
\end{enumerate}

\subsubsection{Data Type}

The first data type is the $cell$. The $cell$ are element of the grid container.

The principal method defined on a cell are:
\begin{description}
 \item[$\FuncCall{SpeedGet}(i)$ :] Witch get the speed at direction $i$. We use staggered grid and consider speed component of a given cell
 to be below left.
  \item[$\FuncCall{SpeedGet}()$ :] Witch get the speed vector.
 \item[$\FuncCall{SpeedSet}(i,val)$ :] Set the speed component $i$ with value $val$.
 \item[$\FuncCall{SpeedSet}(val)$ :] Set the speed vector to value $val$.
 \item[$\FuncCall{AccelerationGet}(i)$ :] Witch get the acceleration at direction $i$.
 \item[$\FuncCall{AccelerationGet}()$ :] Witch get the acceleration vector.
 \item[$\FuncCall{AccelerationSet}(i,val)$ :] Set the acceleration component $i$ with value $val$.
 \item[$\FuncCall{AccelerationSet}(val)$ :] Set the acceleration vector to value $val$.
 \item[$\FuncCall{PressureGet}()$ :] Get the pressure at given cell.
 \item[$\FuncCall{PressureSet}(val)$ :] Set the pressure at given cell to $val$.
 \item[$\FuncCall{GetIsFluid}()$ :] Indicate if the cell is fluid or not.
 \item[$\FuncCall{GetIsAir()}$ :] Indicate if the cell is air or not. This is the opposite of $\FuncCall{GetIsFluid}()$.
 \item[$\FuncCall{FluidSet()}$ :] Set the cell as fluid.
 \item[$\FuncCall{AirSet()}$ :] Set the cell as air.
 \item[$\FuncCall{GetNeighbour}(dir,s)$ :] Get the neighbor cell in direction $dir$ positif or negatif depending on the sign of $s$.
\end{description}

We define the following method for vector:
\begin{description}
 \item[$\FuncCall{Get}(i)$] Get the value at indice $i$ of the vector.
  \item[$\FuncCall{Set}(i,val)$] Set the value at indice $i$ of the vector with value $val$.
\end{description}

The following global variable are used:
\begin{description}
 \item[$viscosity$ :] Is the viscosity used.
 \item[$dt$ :] The time step used.
 \item[$t$ :] The time.
 \item[$h$ :] The spacing.
\end{description}


\subsection{Acceleration evaluation}

To apply gravity we simply add a constant acceleration cf. algorithm \ref{code:ApplyGravity} and algorithm \ref{code:ApplyGravityCell}.
We test if we need to apply gravity with code like algorithm \ref{code:NeedToApplyGravity} which apply gravity to speed component
that touch fluid.

To apply the viscosity we need to calculate the Laplacian and a condition of when apply.
The same condition as for gravity work.

The application is in algorithm \ref{code:ApplyViscosity} the calculation of the Laplacian is in algorithm \ref{code:GetLaplacianSpeed}.

For the convection the application is in algorithm \ref{code:ApplyConvection} the condition are similar than for Gravity.
The convection speed derivative with upwind is done in algorithm \ref{code:GetConvectionElement} and algorithm \ref{code:GetConvectionSpeed}.

\begin{algorithm}
\caption{Algorithm witch test if gravity acceleration need to be applied}
\label{code:NeedToApplyGravity}
\begin{algorithmic}[1]
\Function{NeedToApplyGravity}{cell}
		\If{$cell.\FuncCall{GetIsFluid}()$}
			\State \Return $\True$ \Comment{We are a fluid cell, so the speed component in $y$ is bordering the fluid cell.}
		\EndIf
		\State $cell2 \gets cell.\FuncCall{GetNeighbour}(2,-1)$ 
		\Comment{We now consider the cell below (the gravity is applied in the $y$ direction), to see if we are bordering a fluid cell.}
		\If{$\Not(cell2.\FuncCall{IsValid}())$}
			\State \Return $\False$ \Comment{If the neighbor don't exist we are not near fluid.}
		\EndIf
		\If{$cell2.\FuncCall{GetIsFluid}()$}
			\State \Return $\True$ \Comment{If the neighbor is fluid, we are bordering a fluid cell.}
		\EndIf
		\State \Return $\False$
	\EndFunction
	\end{algorithmic}
\end{algorithm}

\begin{algorithm}
\caption{Algorithm witch apply the gravity acceleration.}
\label{code:ApplyGravity}
\begin{algorithmic}[1]
\Function{ApplyGravity}{cell}
		\If{$\FuncCall{NeedToApplyGravity}(cell)$}
		\State $cell.\FuncCall{AccelerationSet}(2,cell.\FuncCall{AccelerationGet}(2)-g)$ \Comment{Add gravity to the acceleration.}
		\EndIf
\EndFunction
	\end{algorithmic}
\end{algorithm}

\begin{algorithm}
\caption{Algorithm witch apply the gravity acceleration for every cell.}
\label{code:ApplyGravityCell}
\begin{algorithmic}[1]
\Procedure{ApplyGravityCell}{}
		\ForAll{ $cell\in Grid$}
		\State $\FuncCall{ApplyGravity}(cell)$
		\EndFor
\EndProcedure
	\end{algorithmic}
\end{algorithm}

\begin{algorithm}
\caption{Algorithm witch apply the viscosity acceleration.}
\label{code:ApplyViscosity}
\begin{algorithmic}[1]
\Procedure{ApplyViscosity}{}
\ForAll{ $cell\in Grid$}
			\If{$\FuncCall{GetIfApplyViscosity}(it)$}
				\State $it.\FuncCall{AccelerationSet}(it.\FuncCall{AccelerationGet}()+\FuncCall{GetLaplacianSpeed}(it)\cdot viscosity)$
				\Comment{Add the viscosity acceleration to the acceleration.}
			\EndIf
		\EndFor
	\EndProcedure
	\end{algorithmic}
\end{algorithm}

\begin{algorithm}
\caption{Algorithm witch apply the vector Laplacian. The discretization is detailed in section \ref{fixed:vect:laplacian}}
\label{code:GetLaplacianSpeed}
\begin{algorithmic}[1]
\Function{GetLaplacianSpeed}{cell}
	\For{$i=1\To dim$}\Comment{A vector Laplacian consist to apply the Laplacian in all component.}
		\State $temp \gets 0$
			\For{$j=1\To dim$}\Comment{Calculate the Laplacian}
				\State $temp\gets temp-2cell.\FuncCall{SpeedGet}(i)\cdot (h.Get(j))^2$
				\State $temp\gets temp+cell.\FuncCall{GetNeighbour}(j,1).\FuncCall{SpeedGet}(i)\cdot (h.Get(j))^2$
				\State $temp \gets temp+cell.\FuncCall{GetNeighbour}(j,-1).\FuncCall{SpeedGet}(i)\cdot (h.Get(j))^2$
			\EndFor
			\State $ret.\FuncCall{Set}(i,temp)$\Comment{Store the component in the vector $ret$.}
		\EndFor
		\State \Return $ret$ \Comment{\Return the result.}
	\EndFunction
	\end{algorithmic}
\end{algorithm}



\begin{algorithm}
\caption{Algorithm to Apply the convection.}
\label{code:ApplyConvection}
\begin{algorithmic}[1]
\Function{ApplyConvection}{}
		\ForAll{$it \in Grid$}
			\For{$i=1\To dim$}
				\If{$\FuncCall{GetIfApplyConvection}(it,i)$}
					\State $it.\FuncCall{AccelerationSet}(i,it.\FuncCall{AccelerationGet}(i)+\FuncCall{GetConvectionSpeed}(it,i))$ 
					\Comment{We add to the acceleration the convection.}
				\EndIf
			\EndFor
		\EndFor
	\EndFunction
		\end{algorithmic}
\end{algorithm}

\begin{algorithm}
\caption{Algorithm witch calculate $u_{j}\partial_{j}u_{i}$ with upwind scheme. The discretization is detailed in section \ref{fixed:upwind}}
\label{code:GetConvectionElement}
\begin{algorithmic}[1]
\Function {GetConvectionElement}{$i,j,cell$}
		\State $U \gets \FuncCall{GetSpeedAtBound}(cell,i,j)$
		\Comment{Function to get the speed at speed component position $i$ in direction $j$. This can be done by average.}
		\If{$U>0$}
		\Comment{We test if we are upwind or not.}
			\State $U1\gets cell.\FuncCall{SpeedGet}(i)$
			\State $cell \gets cell.\FuncCall{GetNeighbour}(j,-1)$
			\State $U2 \gets cell.\FuncCall{SpeedGet}(i)$
			\State $U \gets U(U1-U2)\cdot h.\FuncCall{Get}(j)$
		\Else
			\State $U1 \gets cell.\FuncCall{SpeedGet}(i)$
			\State $cell \gets cell.\FuncCall{GetNeighbour}(j,1)$
			\State $U2\gets cell.\FuncCall{SpeedGet}(i)$
			\State $U\gets U(U2-U1)\cdot h.\FuncCall{Get}(j)$
		\EndIf
		\State \Return $U$
\EndFunction
			\end{algorithmic}
\end{algorithm}

\begin{algorithm}
\caption{Algorithm witch calculate the convection.}
\label{code:GetConvectionSpeed}
\begin{algorithmic}[1]
 \Function{GetConvectionSpeed}{cell,i}
		\State  $ret \gets 0$
		\For{$j=1\To dim$}
		\Comment{We sum on $j$ of $u_{j}\partial_{j}u_{i}$.}
			\State $ret\gets ret-\FuncCall{GetConvectionElement}(i,j,cell)$
		\EndFor
		\State \Return $ret$
	\EndFunction
 \end{algorithmic}
\end{algorithm}


\subsection{Projection of Acceleration.}

The projection of acceleration is done in following the following step:
\begin{enumerate}
 \item Calculate the divergence of the acceleration ($a$ is the acceleration).
 \begin{equation}
  b=\nabla \cdot a
 \end{equation}
 \item Solve the linear system
 \begin{equation}
  \Delta p=b
 \end{equation}
 \item change the acceleration to:
 \begin{equation}
  a_{new}=a+\nabla p
 \end{equation}
\end{enumerate}

If we have boundary condition, the $b$ vector need to be translated.

The divergence is calculated by algorithm \ref{code:GetDivergence}, the gradient by algorithm \ref{code:GetGradiant}.
A function $\FuncCall{GetIsInDomain}(cell)$ is used, we can consider it similar to $cell.\FuncCall{GetIsFluid}()$.
The pressure is then solved with algorithm \ref{code:ProjectAcceleration}.

\begin{algorithm}
\caption{Algorithm that calculate the divergence. The discretization is detailed in section \ref{fixed:divergence}}
\label{code:GetDivergence}
\begin{algorithmic}[1]
\Function{GetDivergence}{$cell$}
		\State $ret\gets 0$
		\For{$i=1\To dim$}
			\State $ret\gets ret-cell.\FuncCall{SpeedGet}(i)\cdot h.\FuncCall{Get}(i)$
			\State $ret\gets ret+cell.\FuncCall{GetNeighbour}(i,1).\FuncCall{SpeedGet}(i) \cdot h.\FuncCall{Get}(i)$
		\EndFor
		\State \Return $ret$
	\EndFunction
	 \end{algorithmic}
\end{algorithm}

\begin{algorithm}
\caption{Algorithm that calculate the gradient. The discretization is detailed in section \ref{fixed:gradient}.}
\label{code:GetGradiant}
\begin{algorithmic}[1]
\Function{GetGradiant}{$cell,i$}
		\State $ret\gets cell.\FuncCall{PressureGet}()\cdot h.\FuncCall{Get}(i)$
		\State $ret\gets ret-cell.\FuncCall{GetNeighbour}(i,-1).\FuncCall{PressureGet}()\cdot h.\FuncCall{Get}(i)$
		\State \Return $ret$
\EndFunction
		 \end{algorithmic}
\end{algorithm}

\begin{algorithm}
\caption{Algorithm the correction to acceleration to make divergence free.
In this case we assume a case whit 0 Dirichlet boundary condition.}
\label{code:ProjectAcceleration}
\begin{algorithmic}[1]
\Procedure{ProjectAcceleration}{}
		\ForAll{$it\in Grid$}
		\Comment{We calculate the divergence for every cell and store it in a array.
			$it.\FuncCall{ID}()$ is a unique numbering of the array for all cell. }
			\If{$it.\FuncCall{GetIsInDomain}()$}
				\State $b[it.\FuncCall{ID}()] \gets \FuncCall{GetDivergence}(it)$
			\EndIf
		\EndFor
		\State $Grid.\FuncCall{SolveLinear}(b,p)$
		\Comment{We solve the Laplacian system. $p$ is the result and $b$ the second member. The discretization of Laplacian is show in section \ref{fixed:Laplacian}.}
		\ForAll{$it \in Grid$}
			\If{$it.\FuncCall{GetIsInDomain}()$}
			\Comment{If we are in the domain for the Laplacian. We copy the result in the pressure.}
				\State $it.PressureSet(p[it.ID()])$
			\ElsIf
			\Comment{For Dirichlet boundary condition, we set pressure to 0.}
				\State $PressureSet(0)$
			\EndIf
		\EndFor
		\ForAll{$it\in Grid$}
		\Comment{We correct the acceleration.}
			\For{$i=1\To dim$}
				\State $it.\FuncCall{AccelerationSet}(i,it.\FuncCall{AccelerationGet}(i)-\FuncCall{GetGradiant}(it,i))$
			\EndFor
		\EndFor
\EndProcedure
			 \end{algorithmic}
\end{algorithm}

\subsection{Runge-Kutta integration}

To implement the Runge-Kutta method, some new element are needed.
We need to have copy of speed. To do this we add a method $\FuncCall{GetArray}(i)$ witch get the array $i$.
We need to set the default array if no array is precised this is done by function $SetArraySpeed(i)$.
By default if not changer array $0$ is used.
The function $\FuncCall{CalculateAcceleration}()$ calculate the acceleration using the default array.
The only part of code that use explicit array are this one.

The function $\FuncCall{CalculateAcceleration}()$ consist to calculate the acceleration and then do the projection.
This is show in algorithm \ref{code:CalculateAcceleration}.

The Runge-Kutta algorithm for order $4$ is shown in algorithm \ref{code:RungeKutta}.

Depending of the Runge-Kutta scheme choosed the number of copy of speed needed change.

\begin{algorithm}
\caption{Algorithm witch calculate the acceleration.}
\label{code:CalculateAcceleration}
\begin{algorithmic}[1]
 \Procedure{CalculateAcceleration}  {}
	\State $\FuncCall{ApplyConvection}()$
	\State $\FuncCall{ApplyViscosity}()$
	\State $\FuncCall{ApplyGravityCell}()$
	\State $\FuncCall{ProjectAcceleration}()$
	\EndProcedure
 \end{algorithmic}
\end{algorithm}

\begin{algorithm}
\caption{Algorithm that integrate with the Runge Kutta method.}
\label{code:RungeKutta}
\begin{algorithmic}[1]
\Procedure{RungeKutta}{}
\ForAll{$it\in Grid$}
            \State $it.\FuncCall{AccelerationSet}(0)$
            \Comment{We set the acceleration to $0$. So that we can after calculate the acceleration.}
\EndFor
	\State $\FuncCall{CalculateAcceleration}()$
	\Comment{This fonction will add the acceleration to the acceleration array.}
        \ForAll{$it \in Grid$}
            \State $it.\FuncCall{GetArray}(2).\FuncCall{SpeedSet}(it.\FuncCall{GetArray}(0).\FuncCall{SpeedGet}()+it.\FuncCall{AccelerationGet}()\cdot dt \cdot 0.5)$
            \Comment{Array $2$ is set to the speed for the next evaluation. We do not use array $0$ because we need it.}
            \State $it.\FuncCall{GetArray}(1).\FuncCall{SpeedSet}(it.\FuncCall{GetArray}(0).\FuncCall{SpeedGet}()+it.\FuncCall{AccelerationGet}()\cdot \frac{dt}{6})$
            \Comment{Array $1$ is set to the accumulation of the result.}
        \EndFor
        \ForAll{$it \in Grid$}
            \State $it.\FuncCall{AccelerationSet}(0)$
        \EndFor
        \State $\FuncCall{SetArraySpeed}(2)$
        \Comment{The speed for the next evaluation is in array $2$. We set array $2$ as default.}
        \State $t\gets t+0.5\cdot dt$
        \State $\FuncCall{CalculateAcceleration}()$
        \ForAll{$it \in Grid$}
            \State $it.\FuncCall{GetArray}(2).\FuncCall{SpeedSet}(it.\FuncCall{GetArray}(0).\FuncCall{SpeedGet}()+it.\FuncCall{AccelerationGet}()\cdot \frac{dt}{2})$
            \Comment{Array $2$ is set to the speed for the next evaluation. We do not use array $0$ because we need it.}
            \State $it.\FuncCall{GetArray}(1).\FuncCall{SpeedSet}(it.\FuncCall{GetArray}(1).\FuncCall{SpeedGet}()+it.\FuncCall{AccelerationGet}()\cdot \frac{dt}{3})$
	    \Comment{Array $1$ is set to the accumulation of the result.}
       \EndFor
        \ForAll{$it\in Grid$}
            \State $it.\FuncCall{AccelerationSet}(0)$
        \EndFor
        \State $\FuncCall{CalculateAcceleration}()$
        \ForAll{$it \in Grid$}
            \State $it.\FuncCall{GetArray}(2).\FuncCall{SpeedSet}(it.\FuncCall{GetArray}(0).\FuncCall{SpeedGet}()+it.\FuncCall{AccelerationGet}()\cdot dt)$
            \State $it.\FuncCall{GetArray}(1).\FuncCall{SpeedSet}(it.\FuncCall{GetArray}(1).\FuncCall{SpeedGet}()+it.\FuncCall{AccelerationGet}()\cdot \frac{dt}{3})$
        \EndFor
        \ForAll{$it \in Grid$}
            \State $it.\FuncCall{AccelerationSet}(0)$
        \EndFor
        \State $t\gets t+\frac{dt}{2}$
        \State $\FuncCall{CalculateAcceleration}()$
        \ForAll{$it\in Grid$}
            \State $it.\FuncCall{GetArray}(0).\FuncCall{SpeedSet}(it.\FuncCall{GetArray}(1).\FuncCall{SpeedGet}()+it.\FuncCall{AccelerationGet}()\cdot \frac{dt}{6})$
            \Comment{ Array $0$ now contain the new speed.}
         \EndFor
        \State $\FuncCall{SetArraySpeed}(0)$
        \Comment{We set the default array again to $0$ to be ready for the next time step.}
        \EndProcedure
        \end{algorithmic}
\end{algorithm}

\subsection{Complexity}

The complexity is dominated by the $4$ call to $\FuncCall{CalculateAcceleration}()$.
Function $\FuncCall{CalculateAcceleration}()$ then call the function to calculate acceleration.
This function consist for cell where the acceleration apply calculate a term witch depend of the cell itself and his neighbor.
The complexity is then linear on the number of cell.
The complexity for the projection of acceleration. Will depend of the complexity to solve the Laplacian witch depending
on the method used can be linear (multigrid for example is linear). The complexity to calculate the gradient is linear to.

The complexity is then dominated by the complexity to solve the Laplacian.

A more technical complexity is the storage of data on grid. For fixed topology we can use a fixed size array.
But for variable topology we need to choose how we store the data.
To be robust, our method need to be able to do dynamic allocation of memory to be able to represent all possible topology.
Depending the choice made the access time can be constant or $\log(n)$.
The method $\FuncCall{GetNeighbour(dir,s)}()$ was used instead of index vector manipulation because it allow to create storage
with fast access to neighbor cell.

\subsection{Parralelism}

We remark that we can without problem make parallel the $\mathop{\textbf{for all}}$ loop, because we have no dependency:
\begin{itemize}
 \item The calculation of the acceleration don't modify the speed and only read there propper acceleration.
 \item The projection of acceleration when applying the gradient use the already calculated pressure.
 \item The parallelism to solve the linear system will depend on the method used.
\end{itemize}

But technicaly this is only true if the method of manipulation of speed are thread safe.

\subsection{Conclusion}

We have show a working pseudo code that solve the Navier-Stokes equation for fixed boundary condition.
We have shown that the complexity is dominated by the resolution of a Laplacian and can be linear.
To have a real implementation we need to provide some storage related function to access neighbor for example.

This pseudo code is part of the simplified version of the original \textbf{C++} code.
As linear solver Umfpack direct solver was used. Other iteratif library were tester too.
The storage method are constructed around \textbf{C++11} standard library \textbf{Unordered Map} container witch consist of a hash table.

\section{Numerical experiment}

We will now do some numerical experiment in fixed boundary condition to show how it work.

\subsection{Heat Equation}

The more complicated term in the Navier-Stokes equation in grid representation is the convection term, because it's none linear.
\begin{equation}
	(\vect{v}\cdot \vect{\nabla})\vect{v}
\end{equation}

We can have this term 0 everywhere, if we ensure that the direction of variation is orthogonal to the direction of speed.

For example, for a two dimensional problem we work with a flow with a speed only in the $y$ direction. And the value depending only on the position on the $x$ axis.
\begin{equation}
	\vect{v}=\begin{pmatrix}
	         	0\\f(x,t)
	         \end{pmatrix}
\end{equation}
The convection term is then:
\begin{equation}
	(f(x)\partial_{y})\begin{pmatrix}
	         	0\\f(x,t)
	         \end{pmatrix}=0
\end{equation}

The Navier-Stokes equation reduce then to:
\begin{align}
\label{fix:solheat:a}
	\partial_{y}f(x,t)&=0\\
	\frac{\partial f(x,t)}{\partial t}&=F(x,t)+\nu\frac{\partial^2 f(x,t)}{\partial x^2}-\vect{\nabla}p
\end{align}

Equation \ref{fix:solheat:a} is automatically true because we have no dependence on $x$.

The pressure term role is only to cancel non divergence free part coming from the force. Because the part with the viscosity $\nu$ is a linear operator
of a divergence free term.

For a potential force like gravity, the force can be absorbed in the pressure with a proper account of boundary condition.

This equation is an equation of heat.

We consider no external force as gravity.
We will consider as domain $x=[0,2L]$ with fixed speed in $x=0$ and $x=L$,  Neumann boundary condition for pressure in $x=0$ and $x=L$.
Analytically the domain in $y$ expand from $-\infty$ to $\infty$. Numerically, we use Dirichlet boundary condition $y=[0,2L]$ for pressure.
\begin{align*}
    v(x=0,t)&=1\\
    v(x=2L,t)&=-1\\
    v(x,0)&=\begin{cases}
             1&\text{if $x<L$,}\\
             -1& \text{if $x\geq L$.}
            \end{cases}
\end{align*}

\subsubsection{Analytical solution}

$\vect{\nabla}p$ is everywhere 0 because the linear combination of a divergence free field is divergence free.
\begin{equation}
\frac{\partial f(x,t)}{\partial t}=\nu\frac{\partial^2 f(x,t)}{\partial x^2}
\end{equation}

We first look for 0 boundary condition. We will denote this case by $f_0$
We first use as ansatz that we can separate variable:
\begin{equation}
 f_0(x,t)=X(x)T(t)
\end{equation}

The equation then give:
\begin{equation}
	\frac{\dot{T}}{\nu T}=\frac{X''}{X}
\end{equation}

Because the rhs only depend on $x$ and the lhs only depend on $t$. This need to depend only on a constant value $-\lambda$.

This give use two uncoupled equation depending on $\lambda$.

\begin{align}
\dot{T}&=-\lambda \nu T\\
X''&=-\lambda X
\end{align}

The solution for $T$ is:
\begin{equation}
  T\propto e^{-\lambda \nu t}
\end{equation}

The solution for $X$ is:
\begin{equation}
	X\propto \sin(\sqrt{\lambda} x)
\end{equation}

We only have $\sin$ because $\sin(0)=0$.
Because we ask $\sin(2L)=0$.
\begin{equation}
	0=\sin(2\sqrt{\lambda} L)=\sin\left(n\frac{\pi}{2L}\right)
\end{equation}

This then give the condition on $\lambda$:
\begin{equation}
	\sqrt{\lambda}=n\frac{\pi}{2L}
\end{equation}

A linear combination of this for different $n$ is solution too.
We found the good linear combination from the Fourier series of the initial condition $f(0,t)$.

\begin{align}
  f_0(x,t)&=\sum_{n=1}^{\infty}C_{n}\sin\left(\frac{n\pi}{2L}x\right)e^{-\frac{n^2\pi^2\nu t}{4L^2}}\\
  C_{n}&=\frac{1}{L}\int_{0}^{2L}f(x,0)\sin\left(\frac{n\pi}{2L}x\right) dx
\end{align}

For a none 0 boundary condition, we need to add a particular solution, so that $f_p(0,0)=1$ and $f_p(2L,0)=-1$.
\begin{equation}
	f_{p}(x,t)=-\frac{x-L}{L}
\end{equation}

$f_{0}(x,0)$ is then given by:
\begin{equation}
  f_{0}(x,0)=f(x,0)-f_{p}(x,t)=\begin{cases}
                               	\frac{x}{L}&\text{if $x<L$},\\
                               	\frac{x-L}{L}-1&\text{if $x\geq L$.}
                               \end{cases}
\end{equation}

The general solution is then:
\begin{align}
  f(x,t)&=-\frac{x-L}{L}+\sum_{n=1}^{\infty}C_{n}\sin\left(\frac{n\pi}{2L}x\right)e^{-\frac{n^2\pi^2\nu t}{4L^2}}\\
  C_{n}&=\frac{1}{L}\int_{0}^{2L}f(x,0)\sin\left(\frac{n\pi}{2L}x\right) dx
\end{align}

This is plotted for $2L=\unit{0.1}{\metre}$ and $\nu=\unit{1.307\cdot 10^{-6}}{\pascal\second}$ witch is the viscosity of Water for $\unit{10}{\degreecelsius}$ and for $n_{max}=1'000'000$ in figure \ref{fixed:solheat:fig:sol}.

Note that the fact that we have truncated the sum can be considered as a slightly change of initial condition, and an analytical solution of this solution.
Numerically the difference is of the level of the numerical precision.



The analytical solution show that the effect of viscosity is to smooth the speed distribution. The time needed to do that is of $\unit{500}{\second}$
witch is a very long time with respect of the distance traveled in this time ($\unit{500}{\meter}$).

But this doesn't say that viscosity is not important, because it's effect is inversely proportional squared of $L$.

\subsection{}

\subsubsection{Numerical solution}

We now use our numerical program to solve the same problem. The effective problem that we will solve is the heat equation with as solver Runge-Kutta.
The other part of the program as convection and projection are done but should have a negligible effect.

The solution for different time $t$ are shown in figure \ref{fix:solheat:fig:num1}, \ref{fix:solheat:fig:num100}, \ref{fix:solheat:fig:num300} ,\ref{fix:solheat:fig:num500}. We have an excellent agreement with the analytical result.
For $dt=2\second$ for the Euler method and $dt=\unit{3}{\second}$ for the Runge-Kutta method we have divergence (result completely wrong).
Note that the time step used is bigger than the CFL condition that is $dt_{max}=hu_{max}=\unit{0.002}{\second}$. This can be done because the viscosity is small and the none linear term cancel.

The solution with low resolution for different time $t$ are shown in figure \ref{fix:solheat:fig:num1b}, \ref{fix:solheat:fig:num100b}, \ref{fix:solheat:fig:num300b} ,\ref{fix:solheat:fig:num500b}.
We are in excellent agreement with this low resolution. The reason can be because the equation is linear. 



\subsection{Irrotational}

\subsubsection{Analytical Solution}

We will be interested to a solution without gravity and viscosity.


We can obtain a divergence free solution witch is Irrotational from the gradient of a function.
With:
\begin{align}
	\phi&=(x^2-y^2)t\\
	\Delta \phi &=2t(1-1)=0\\
	\vect{v}&=\nabla \phi=2t\begin{pmatrix}
	               	x\\
	               	-y
	               \end{pmatrix}
\end{align}

The convection term is:
\begin{align}
	\partial_{x}\vect{v}&=2t\begin{pmatrix}
	                        	1\\
	                        	0
	                        \end{pmatrix}\\
	\partial_{y}\vect{v}&=\partial_{y}\vect{v}&=2t\begin{pmatrix}
	                        	0\\
	                        	-1
	                        \end{pmatrix}\\
 \left(\vect{v}\cdot\right) \nabla \vect{v}&=4t^2\begin{pmatrix}
                                                 	x\\
                                                 	y
                                                 \end{pmatrix}
\end{align}

This is a gradient of a function. So it's irrotational. $\dot{\vect{v}}$ is irrotational too.
So the difference of the two is irrotational too and lead to the pressure gradient.

This gave the following result:
\begin{align}
\dot{\vect{v}}+ \left(\vect{v}\cdot\right) \nabla \vect{v}&=-\nabla p\\
\dot{\vect{v}}&=2\begin{pmatrix}
	               	x\\
	               	-y
	               \end{pmatrix}\\
\left(\vect{v}\cdot\right) \nabla \vect{v}&=4t^2\begin{pmatrix}
                                                 	x\\
                                                 	y
                                                 \end{pmatrix}\\
-\nabla p=\begin{pmatrix}
           -2x-4t^2x\\
           2y-4t^2y
           \end{pmatrix}\\
\end{align}

This solution is analytical.

For the numerical problem the following was done.
For not having to implement none 0 Neumann boundary condition in the projection of speed as shown in the section on projection.
I have implemented the problem like this:
\begin{align}
	\dot{\vect{v}}+ \left(\vect{v}\cdot\right) \nabla \vect{v}&=-\nabla p+\nabla F\\
\end{align}

Where $\nabla F$ is taken to be the analytical solution for $-\nabla p$.
$\nabla p$ is then calculated with Neumann boundary condition and should be 0 to match analytical solution.

The value at the boundary for the speed are found from the analytical solution.

This is a little an artificial numerical problem but is interesting because,
it mainly test that the numerical sheme for convection doesn't add too much rotational component
where an analytical irrotational one is expected. If the value of the gradient component is wrong, is not a problem
because it will be changed by the projection.

The time step was chosen by the CFL condition given by:
\begin{equation}
  dt=\alpha \frac{h}{v_{max}}
\end{equation}

Where $h$ is the spatial spacing and $v_{max}$ the maximal norm of speed in the domain.
$\alpha$ is a parameter witch normally need to be smaller or equal to 1.

The condition consist of what portion of a cell you are allowed to traverse in a given time step.

We will test 3 parameter:
\begin{itemize}
	\item The CFL factor, with two value, 1 and 10.
	\item The resolution with $N=10$ and $n=50$.
	\item The integration method with Euler and Runge-Kutta.
\end{itemize}
 
The first thing to note is that a CFL factor of 10 will lead to a numerical blow up.
The blow up doesn't happen immediately because we use a time step limiter to avoid that very small speed lead to very big time step.

We note that the region where the difference come first are high speed region because they break the condition first.
This can be seen in figure \ref{fix:comp_10_11_1}.

The space spacing doesn't change the problem, it will happen at another time because of the time step limiter.
But it will happen.

Runge-Kutta method break before Euler method.

When the CFL condition is respected we have an error of the order of $10^{-7}$ witch seem to be of the same amount than the global numerical error.

We have a none 0 in the comparison in the boundary because of the numerical error coming from output to file of the result,
read of the file and comparison done in python (another language than where the main code was written (c++) ).

Here is the tabular of all figure for $alpha=10$. The first column for a given time step are $x$ and $y$ figure.
The second are comparison with respect to the analytical solution.

\begin{tabular}{|l|l|l|}
	\hline
Integration method& $N=50$ & $N=10$\\
\hline
Euler &\begin{tabular}{lcc}
	$\unit{1.048}{\second}$&\ref{fix:plot_10_11_1} \ref{fix:plot_10_11_2}&\ref{fix:comp_10_11_1} \ref{fix:comp_10_11_2}\\
	$\unit{1.36}{\second}$&\ref{fix:plot_10_16_1} \ref{fix:plot_10_16_2}&\ref{fix:comp_10_16_1} \ref{fix:comp_10_16_2}\\
	$\unit{1.40}{\second}$&\ref{fix:plot_10_20_1} \ref{fix:plot_10_20_2}&\ref{fix:comp_10_20_1} \ref{fix:comp_10_20_2}\\
	\end{tabular}&\begin{tabular}{lcc}
	$\unit{1.00}{\second}$&\ref{fix:plot_11_10_1} \ref{fix:plot_11_10_2}&\ref{fix:comp_11_10_1} \ref{fix:comp_11_10_2}\\
	$\unit{2.00}{\second}$&\ref{fix:plot_11_20_1} \ref{fix:plot_11_20_2}&\ref{fix:comp_11_20_1} \ref{fix:comp_11_20_2}\\
	$\unit{2.70}{\second}$&\ref{fix:plot_11_27_1} \ref{fix:plot_11_27_2}&\ref{fix:comp_11_27_1} \ref{fix:comp_11_27_2}\\
	$\unit{2.85}{\second}$&\ref{fix:plot_11_32_1} \ref{fix:plot_11_32_2}&\ref{fix:comp_11_32_1} \ref{fix:comp_11_32_2}\\
	\end{tabular} \\
	\hline
Runge-Kutta &\begin{tabular}{lcc}
	$\unit{0.70}{\second}$&\ref{fix:plot_12_7_1} \ref{fix:plot_12_7_2}&\ref{fix:comp_12_7_1} \ref{fix:comp_12_7_2}\\
	$\unit{0.77}{\second}$&\ref{fix:plot_12_8_1} \ref{fix:plot_12_8_2}&\ref{fix:comp_12_8_1} \ref{fix:comp_12_8_2}\\
	\end{tabular} &\begin{tabular}{lcc}
	$\unit{1.00}{\second}$&\ref{fix:plot_13_10_1} \ref{fix:plot_13_10_2}&\ref{fix:comp_13_10_1} \ref{fix:comp_13_10_2}\\
	$\unit{2.00}{\second}$&\ref{fix:plot_13_20_1} \ref{fix:plot_13_20_2}&\ref{fix:comp_13_20_1} \ref{fix:comp_13_20_2}\\
	$\unit{2.10}{\second}$&\ref{fix:plot_13_21_1} \ref{fix:plot_13_21_2}&\ref{fix:comp_13_21_1} \ref{fix:comp_13_21_2}\\
	\end{tabular}\\
	\hline
\end{tabular}

For $\alpha=1$.

\begin{tabular}{|l|l|l|}
	\hline
Integration method& $N=50$ & $N=10$\\
\hline
Euler &\begin{tabular}{lcc}
	$\unit{1.00}{\second}$&\ref{fix:plot_14_68_1} \ref{fix:plot_14_68_2}&\ref{fix:comp_14_68_1} \ref{fix:comp_14_68_2}\\
	$\unit{2.00}{\second}$&\ref{fix:plot_14_274_1} \ref{fix:plot_14_274_2}&\ref{fix:comp_14_274_1} \ref{fix:comp_14_274_2}\\
	$\unit{5.00}{\second}$&\ref{fix:plot_14_1713_1} \ref{fix:plot_14_1713_2}&\ref{fix:comp_14_1713_1} \ref{fix:comp_14_1713_2}\\
	\end{tabular}&\begin{tabular}{lcc}
	$\unit{1.01}{\second}$&\ref{fix:plot_15_14_1} \ref{fix:plot_15_14_2}&\ref{fix:comp_15_14_1} \ref{fix:comp_15_14_2}\\
	$\unit{2.01}{\second}$&\ref{fix:plot_15_50_1} \ref{fix:plot_15_50_2}&\ref{fix:comp_15_50_1} \ref{fix:comp_15_50_2}\\
	$\unit{5.01}{\second}$&\ref{fix:plot_15_302_1} \ref{fix:plot_15_302_2}&\ref{fix:comp_15_302_1} \ref{fix:comp_15_302_2}\\
	\end{tabular} \\
	\hline
Runge-Kutta &\begin{tabular}{lcc}
	$\unit{1.00}{\second}$&\ref{fix:plot_16_68_1} \ref{fix:plot_16_68_2}&\ref{fix:comp_16_68_1} \ref{fix:comp_16_68_2}\\
	$\unit{2.00}{\second}$&\ref{fix:plot_16_274_1} \ref{fix:plot_16_274_2}&\ref{fix:comp_16_274_1} \ref{fix:comp_16_274_2}\\
	$\unit{5.00}{\second}$&\ref{fix:plot_16_1713_1} \ref{fix:plot_16_1713_2}&\ref{fix:comp_16_1713_1} \ref{fix:comp_16_1713_2}\\
	\end{tabular} &\begin{tabular}{lcc}
	$\unit{1.02}{\second}$&\ref{fix:plot_17_14_1} \ref{fix:plot_17_14_2}&\ref{fix:comp_17_14_1} \ref{fix:comp_17_14_2}\\
	$\unit{2.00}{\second}$&\ref{fix:plot_17_50_1} \ref{fix:plot_17_50_2}&\ref{fix:comp_17_50_1} \ref{fix:comp_17_50_2}\\
	$\unit{5.01}{\second}$&\ref{fix:plot_17_302_1} \ref{fix:plot_17_302_2}&\ref{fix:comp_17_302_1} \ref{fix:comp_17_302_2}\\
	\end{tabular}\\
	\hline
\end{tabular}


