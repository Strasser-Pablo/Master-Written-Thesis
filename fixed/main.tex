\chapter{Fixed}
\minitoc
\section{Introduction}

We will in this chapter be interested in the case of fixed domain Navier-Stokes with grid discretisation.
This case is easier than the variable domain case and serve as basis to more complicated case.

\section{Overview}

To solve the Navier-Stokes equation, we will cut the equation in two step.
\begin{enumerate}
 \item Evolve speed without consideration of pressure and divergence free.
  \begin{equation}
   \partial_t \vect{v}(\vect{x} ,t)+(\vect{v}(\vect{x},t)\cdot\vect{\nabla} ) \vect{v}(\vect{x} ,t)=\frac{\vect{F}(\vect{x},t)}{\rho(\vect{x},t)}+\nu \Delta \vect{u}(\vect{x},t)
  \end{equation}
 \item We project the none divergence free speed found before to a divergence free speed.
  \begin{align}
   \vect{v}_{new}&=v-\vect{\nabla}\phi\\
   \Delta \phi&=\vect{\nabla} \cdot \vect{v}
  \end{align}
\end{enumerate}

After this step the discrete evaluation of divergence will be divergence free exactly (to numerical precision) as long as the shame used respect:
\begin{equation}
 \Delta=\vect{\nabla}\cdot \vect{\nabla}
\end{equation}

\begin{rem}
 This method to have a divergence free speed is different that what we will find if we use the analytical expression for $p$ found in equation \ref{analytical:pressure}:
 \begin{equation}
  \Delta p=-\rho \vect{\nabla} \cdot (\vect{v}\cdot \vect{\nabla}) \vect{v}
 \end{equation}
 
 But using this equation, will not ensure in the numerical case that the speed is divergence free.
 The latter equation will not be used because we consider that being divergence free at every time step is more important than
 to have the good equation for pressure but where time discretisation error will create more divergence.
\end{rem}

\section{Evolve in time}

We need to solve:
  \begin{equation}
   \partial_t \vect{v}(\vect{x} ,t)=\frac{\vect{F}(\vect{x},t)}{\rho(\vect{x},t)}+\nu \Delta \vect{u}(\vect{x},t)-(\vect{v}(\vect{x},t)\cdot\vect{\nabla} ) \vect{v}(\vect{x} ,t)
  \end{equation}
  
  \subsection{Analytical solution to Burger's equation}
  This equation is know as the Burger equation and can be solved analytically for $\nu=0$.
  The solution is given by the method of characteristics. Using the same method that we us in section \ref{analytical:convectif}.
  \begin{equation}
   \frac{d u_{\lambda}(t)}{dt}=\frac{\vect{F}(\vect{x},t)}{\rho(\vect{x},t)}
  \end{equation}
  
  The change of speed is given by the force.
  
  This equation can after a finite time have more that one solution and don't have continuous and derivable solution everywhere.
  
  We see this in taking for example a sinus as input speed for the case without force.
  The high speed region of the sinus will go faster than the slow point of the sinus.
  After a given time we will find a multi-valued solution. This is the sign, that we need to consider discontinuous solution after this time.
  But solution are no more unique, other condition need to be added.
  
  For the 3d case of Navier-Stokes equation we do not know if we have unicity after a given time.
  We know that we have a solution but not how many.
  
  In discretized version, we only need to solve for one time step.
  The projection step will change the speed everywhere.

  \subsection{Method of Line}
  
  A typical method of resolution is to discretize in time and space separately.
  We consider that the Burger equation is given by:
  \begin{align}
   \label{fixed:lineA}
   \partial_t \vect{v}(\vect{x} ,t)&=f(\vect{v}(\vect{x}),\vect{x})\\
   \label{fixed:lineB}
   f(t,\vect{v}(\vect{x}),\vect{x})&=\frac{\vect{F}(\vect{x},t)}{\rho(\vect{x},t)}+\nu \Delta \vect{u}(\vect{x},t)
  \end{align}
  
Equation \ref{fixed:lineA} is now an ordinary differential equation for a field $\vect{v}$.
Equation \ref{fixed:lineB} is the evaluation of an expression with derivative at a given time and position.

\subsubsection{Runge-Kutta}

A Runge-Kutta method is a method to solve the ODE (I drop the $\vect{x}$ dependence, but it's understood that this equation is for a vector formed in concatenating every speed):
\begin{equation}
 \partial_t \vect{v}(t)=f(\vect{v})
\end{equation}

The solution at time $n+1$ is given from the solution at time $n$ with $\Delta t$ the time step and $a$, $b$ parameter of the method.
\begin{align*}
	\vect{v}_{n+1}&=\vect{v}_{n}+\sum_{i=1}^{s}b_{i}k_{i}\\
	k_{j}&=\Delta t f(\vect{v}_{n}+\sum_{i=1}^{j-1}a_{j,i}k_{i})
\end{align*}

\subsubsection{Spatial Evaluation}
To apply the Method of Line, we need to evaluate derivative.

\section{Projection}
\label{fixed:projection}
The projection step consist to apply:
 \begin{align}
  \vect{v}_{new}&=v-\vect{\nabla}\phi\\
  \Delta \phi&=\vect{\nabla} \cdot \vect{v}\label{fixed:project}
 \end{align}

 Equation \ref{fixed:project} typically consist to solve a linear system.
 
 \section{Runge-Kutta and Projection}
 
 A possible idea, is to project after every evaluation of speed int the Runge-Kutta method.
 We first define the projection for a vector $\vect{v}$, $P(\vect{v})$.
 Using \ref{fixed:projection} $P(\vect{v})$ can be written:
 \begin{equation}
  P(\vect{v})=-\vect{\nabla}\left(\vect{\nabla}\cdot \vect{\nabla}\right)^{-1}\vect{\nabla}\cdot\vect{v}
 \end{equation}

 $P$ is a linear operator because derivative are linear operator.
 
 From the Navier-Stokes equation:
 \begin{equation}
   \partial_t \vect{v}(\vect{x} ,t)=-(\vect{v}(\vect{x},t)\cdot\vect{\nabla} ) \vect{v}(\vect{x} ,t)+\frac{\vect{F}(\vect{x},t)}{\rho(\vect{x},t)}+\nu \Delta \vect{u}(\vect{x},t)-\nabla \phi
 \end{equation}

 Can be rewritten:
 \begin{align}
 \label{fixed:rung-projA}
  \partial_t \vect{v}(\vect{x} ,t)&=f(\vect{v})-\nabla \phi \\
  \intertext{With:}
  f(\vect{v})&=-(\vect{v}(\vect{x},t)\cdot\vect{\nabla} ) \vect{v}(\vect{x} ,t)+\frac{\vect{F}(\vect{x},t)}{\rho(\vect{x},t)}+\nu \Delta \vect{u}(\vect{x},t)
 \end{align}
 
 Taking the divergence of \ref{fixed:rung-projA} we can determine $\phi$.
 \begin{equation}
  \Delta \phi=\nabla \cdot f(\vect{v})
 \end{equation}

 The gradient of $\phi$ is then:
 
 \begin{equation}
  -\vect{\nabla}\phi=P(f(\vect{v}))
 \end{equation}

 The Navier-Stokes equation can then be rewritten as:
 \begin{equation}\label{fixed:simpl_navier}
 \partial_t \vect{v}(\vect{x} ,t)=f(\vect{v})+P(f(\vect{v})) 
\end{equation}

This equation is analytically exact.

We would expect that solving this equation with a Runge-Kutta algorithm will lead to a numerical error in Divergence.

To show that this is not the case, we show that another method to force divergence free of every speed is in fact equivalent to this one.

We define the following scheme.
\begin{align}
\vect{\tilde{v}}_{n+1}&=\vect{\tilde{v}}_{n}+\sum_{i=1}^{s}b_{i}\tilde{k}_{i}+P\left(\vect{\tilde{v}}_{n}+\sum_{i=1}^{s}b_{i}\tilde{k}_{i}\right)\\
\tilde{k}_{i}&=\Delta t f\left(\vect{\tilde{v}}_{n}+\sum_{j=1}^{i-1}a_{ij}\tilde{k}_{j}+P\left(\vect{\tilde{v}}_{n}+\sum_{j=1}^{i-1}a_{ij}\tilde{k}_{j}\right)\right)
\end{align}

We project at every estimation of speed. Making by construction every speed divergence free.

We now write the Runge-Kutta expression for \ref{fixed:simpl_navier}.

\begin{align}
	\vect{v}_{n+1}&=\vect{v}_{n}+\sum_{i=1}^{s}b_{i}k_{i}\\
	k_{i}&=\Delta t f\left(\vect{v}_{n}+\sum_{j=1}^{i-1}a_{ij}k_{j}\right)+\Delta t P\left(f\left(\vect{v}_{n}+\sum_{j=1}^{i-1}a_{ij}k_{j}\right)\right)
\end{align}

\begin{thm}
If $\vect{v}_{n}=\vect{\tilde{v}}_{n}$ and $\vect{v}_n$ is divergence free then $\vect{v}_{n+1}=\vect{\tilde{v}}_{n+1}$ 
\end{thm}
 \begin{proof}
 We begin by prove the following lemma.
 \begin{lem}
 \begin{equation}
  k_{i}=\tilde{k}_{i}+P(\tilde{k}_{i})
 \end{equation}
 \end{lem}
 \begin{proof}
 By recurrence on $i$.
 We begin with $i=1$:
 \begin{equation}
  k_{1}=\Delta tf\left(\vect{v}_n\right)+\Delta tP\left(f\left(\vect{v}_n\right)\right)
 \end{equation}
\begin{equation}
 \tilde{k}_{1}=\Delta tf\left(\vect{v}_n\right)
\end{equation}
 \begin{equation}
  k_{1}=\Delta t f\left(\vect{v}_n\right)+\Delta tP\left(f\left(\vect{v}_n\right)\right)=\tilde{k}_1+P\left(\tilde{k}_1\right)
 \end{equation}
 
 Assuming true for smaller $i$:
 \begin{align*}
  k_{i}&=\Delta tf\left(\vect{v}_n+\sum_{j=1}^{n}a_{ij}k_{j}\right)+\Delta tP\left(f\left(\vect{v}_n+\sum_{j=1}^{i-1}a_{ij}k_{ij}\right)\right)\\
  &=\Delta tf\left(\vect{v}_n+\sum_{j=1}^{i-1}a_{ij}\left(\tilde{k}_{j}+P\left(\tilde{k}_{j}\right)\right)\right)+\Delta tP\left(f\left(\vect{v}_n+\sum_{j=1}^{i-1}a_{ij}\left(\tilde{k}_{j}+P\left(\tilde{k}_{j}\right)\right)\right)\right)\\
  \tilde{k}_{i}&=\Delta tf\left(\vect{v}_n+\sum_{j=1}^{i-1}a_{ij}\tilde{k}_{j}+P\left(\vect{v}_n+\sum_{j=1}^{i-1}a_{ij}\tilde{k}_{j}\right)\right)\\
  &=\Delta tf\left(\vect{v}_n+\sum_{j=1}^{i-1}a_{ij}\left(\tilde{k}_{j}+P\left(\tilde{k}_{j}\right)\right)\right)\\
  k_{i}&=\tilde{k}_{i}+P\left(\tilde{k}_{i}\right)
 \end{align*}
 \end{proof}
 
 We now write the expression for $\vect{v}_{n+1}$ and $\vect{\tilde{v}}_{n+1}$ using the lemma.
 
 \begin{align*}
 \vect{v}_{n+1}&=\vect{v}_{n}+\sum_{i=1}^{s}b_{i}k_{i}\\
 &=\vect{v}_{n}+\sum_{i=1}^{s}b_{i}\left(\tilde{k}_{i}+P\left(\tilde{k}_{i}\right)\right)\\
 \vect{\tilde{v}}_{n+1}&=\vect{\tilde{v}}_{n}+\sum_{i=1}^{s}b_{i}\tilde{k}_{i}+P\left(\vect{\tilde{v}}_{n}+\sum_{i=1}^{s}b_{i}\tilde{k}_{i}\right)\\
 &=\vect{\tilde{v}}_{n}+\sum_{i=1}^{s}b_{i}\left(\tilde{k}_{i}+P\left(\tilde{k}_{i}\right)\right)\\
 \intertext{Using that $\vect{v}_{n}=\tilde{\vect{v}_{n}}$}
 \vect{v}_{n+1}&=\vect{\tilde{v}}_{n+1}
 \end{align*}

 \end{proof}
 
 \begin{thm}
 For the implicit Runge-Kutta case, where the sum to $i-i$ go now to $s$.
If $\vect{v}_{n}=\vect{\tilde{v}}_{n}$ and $\vect{v}_n$ is divergence free then $\vect{v}_{n+1}=\vect{\tilde{v}}_{n+1}$ 
\end{thm}
 \begin{proof}
 We begin by prove the following lemma.
 \begin{lem}
 \begin{equation}
  k_{i}=\tilde{k}_{i}+P(\tilde{k}_{i})
 \end{equation}
 \end{lem}
 \begin{proof}
Because the sum goes to $s$ and not $i-1$ we cannot use recurrence.
We now will use vector notation for $k$.

We construct $k$ and $\tilde{k}$ vector with the vector $k_{i}$ and $\tilde{k}_{i}$
\begin{align}
 k&=\begin{pmatrix}
    k_{1}\\
    \vdots\\
    k_{s}\\
   \end{pmatrix}\\
 \tilde{k}&=\begin{pmatrix}
    \tilde{k}_{1}\\
    \vdots\\
    \tilde{k}_{s}\\
   \end{pmatrix}
\end{align}

We form $\hat{P}$ and $\hat{f}$ from a Kronecker product:
\begin{align}
 \hat{P}&=\eye_{s}\kron P=\begin{pmatrix}P	&\ldots	&0\\
			\vdots &\ddots 	&\vdots\\
			0	&0	&P\\
         \end{pmatrix}\\
 \hat{f}&=\eye_{s}\kron f\begin{pmatrix}f	&\ldots	&0\\
			\vdots &\ddots 	&\vdots\\
			0	&0	&f\\
         \end{pmatrix}
\end{align}

We now define $\hat{v}_n$:
\begin{equation}
 \hat{v}_{n}=\begin{pmatrix}
               v_{n}\\
               \vdots\\
               v_{n}
              \end{pmatrix}
\end{equation}

We now define $A$ with the following Kronecker product:
\begin{equation}
 A=\begin{pmatrix}
    a_{11}	&\ldots	&a_{1s}\\
    \vdots	&\ddots	&\vdots\\
    a_{s1}	&\ldots	&a_{ss}\\
   \end{pmatrix} \kron \eye
\end{equation}

$A$ and $\hat{P}$ are commutative, because of the mixed-product property.
\begin{align}
A\hat{P}&=\left(\begin{pmatrix}
    a_{11}	&\ldots	&a_{1s}\\
    \vdots	&\ddots	&\vdots\\
    a_{s1}	&\ldots	&a_{ss}\\
   \end{pmatrix} \kron \eye \right)
   \left(\eye_{s}\kron P\right)=
         \begin{pmatrix}
    a_{11}	&\ldots	&a_{1s}\\
    \vdots	&\ddots	&\vdots\\
    a_{s1}	&\ldots	&a_{ss}\\
    \end{pmatrix}\kron P\\
    %
    \hat{P}A&=
   \left(\eye_{s}\kron P\right)
         \left(\begin{pmatrix}
    a_{11}	&\ldots	&a_{1s}\\
    \vdots	&\ddots	&\vdots\\
    a_{s1}	&\ldots	&a_{ss}\\
   \end{pmatrix} \kron \eye \right)=
         \begin{pmatrix}
    a_{11}	&\ldots	&a_{1s}\\
    \vdots	&\ddots	&\vdots\\
    a_{s1}	&\ldots	&a_{ss}\\
    \end{pmatrix}\kron P
\end{align}

\begin{align}
 k&=(1+\hat{P})\Delta t\hat{f}(\hat{v}_{n}+Ak)\\
 \tilde{k}&=\Delta t \hat{f}((1+\hat{P})(\hat{v}_{n}+Ak))
\end{align}

We begin to define 2 functions:
\begin{align}
 k&=g(k)\\
 \tilde{k}&=\tilde{g}(\tilde{k})\\
 g(x)&=(1+\hat{P})\Delta t \hat{f}(\hat{v}_{n}+Ax)\\
 \tilde{g}(x)&=\Delta t \hat{f}((1+\hat{P})(\hat{v}_{n}+Ax))
\end{align}

$k$ and $\tilde{k}$ are fixed point of function $g$ and $\tilde{g}$.

We now calculate:
\begin{align}
 (1+\hat{P})\tilde{g}(x)&=(1+\hat{P})\Delta t \hat{f}((1+\hat{P})(\hat{v}_{n}+Ax))=(1+\hat{P})\Delta t \hat{f}(\hat{v}_{n}+A (1+\hat{P})x)\\
 &=g((1+\hat{P})x)
\end{align}

We now substitute $g$ in $\tilde{g}$
\begin{equation}
(1+\hat{P})\tilde{k}=g((1+\hat{P})\tilde{k})
\end{equation}

$(1+\hat{P})\tilde{k}$ is a fixed point of $g$.
But fixed point of $g$ is $k$.

\begin{equation}
 k=(1+\hat{P})\tilde{k}
\end{equation}
This is exactly the vector version of what we wanted to prove.
 \end{proof}
 
 The rest is exactly the same than in the first case:
 
 We now write the expression for $\vect{v}_{n+1}$ and $\vect{\tilde{v}}_{n+1}$ using the lemma.
 
 \begin{align*}
 \vect{v}_{n+1}&=\vect{v}_{n}+\sum_{i=1}^{s}b_{i}k_{i}\\
 &=\vect{v}_{n}+\sum_{i=1}^{s}b_{i}\left(\tilde{k}_{i}+P\left(\tilde{k}_{i}\right)\right)\\
 \vect{\tilde{v}}_{n+1}&=\vect{\tilde{v}}_{n}+\sum_{i=1}^{s}b_{i}\tilde{k}_{i}+P\left(\vect{\tilde{v}}_{n}+\sum_{i=1}^{s}b_{i}\tilde{k}_{i}\right)\\
 &=\vect{\tilde{v}}_{n}+\sum_{i=1}^{s}b_{i}\left(\tilde{k}_{i}+P\left(\tilde{k}_{i}\right)\right)\\
 \intertext{Using that $\vect{v}_{n}=\tilde{\vect{v}_{n}}$}
 \vect{v}_{n+1}&=\vect{\tilde{v}}_{n+1}
 \end{align*}

 \end{proof}
 
 \begin{cor}
Solving Navier-Stokes equation with the analytical good choice of pressure.
Is exactly divergence free at every approximation of speed.
 \end{cor}
 \begin{proof}
  The expression of $v$ is the solution of the Navier-Stokes equation with the good choice of pressure.
  $\tilde{v}$ is the solution of Runge-Kutta equation without pressure witch is projected at every speed estimation.
  This is divergence free.
  $v$ and $\tilde{v}$ are the same. With end the proof.
 \end{proof}
 
 \begin{cor}
  Using the expression for $v$ or $\tilde{v}$ we are $k$ order precise in time. Where $k$ is the order of the Runge-Kutta method.
 \end{cor}
 \begin{proof}
  $v$ and $\tilde{v}$ are the same.
  $v$ consist only of a Runge-Kutta method with a good chosen function.
  So the order in time of $v$ and $\tilde{v}$ is the same that the method order.
 \end{proof}

