\chapter{Notation and basic property}
\minitoc


\section{Vector}

We denote a vector with the dimension of physical space (2d or 3d) by $\vect{v}$.
We only use Cartesian coordinate systems.
We indicate a given component of a vector with indices, e.g. $\vect{v}_1$ for the first component.

\section{Differential operator}

We denote derivatives by  $\partial_i$, where $i$ is the spatial component with respect to which we take the derivative,
\begin{equation}
 \partial_i v(\vect{x})=\frac{\partial v(\vect{x})}{\partial \vect{x}_{i}}.
\end{equation}

\begin{definition}[Nabla]
 In Cartesian coordinates, the $\vect{\nabla}$ is a vector that cannot commute (e.g. $\nabla \cdot \vect{v}\neq \vect{v}\cdot \nabla$).
 Its expression is given by:
 \begin{equation}
 \vect{\nabla}_{i}=\partial_{i}.
 \end{equation}
\end{definition}

With the $\vect{\nabla}$ we can form the following expression.
With $n$ the dimension of the vector:
\begin{equation}
 \vect{\nabla}p=\begin{pmatrix}\partial_1 p\\ \vdots \\ \partial_n p\end{pmatrix},
\end{equation}
\begin{equation}
 \vect{\nabla}\cdot \vect{v}=\sum_{i=1}^{n}\partial_i \vect{v}_i,
\end{equation}
\begin{equation}
 \left((\vect{v}\cdot\vect{\nabla})\vect{v}\right)_{i}=\sum_{j=1}^{n}v_{j}\partial_{j} v_{i}.
\end{equation}


\begin{definition}[Laplacian]
The Laplacian $\Delta$ consists of a scalar operator:
\begin{equation}
\Delta p=\vect{\nabla}\cdot \vect{\nabla}p=\sum_{i=1}^{n} \partial_{i}\partial_{i}p.
\end{equation}
\end{definition}

\begin{definition}[Vector Laplacian]
A vector version of the Laplacian consists of applying the Laplacian to all components of the vector.
\begin{equation}
 \Delta \vect{v}=\begin{pmatrix} \Delta v_{1}\\
                  \vdots\\ \Delta v_{n}
                 \end{pmatrix}.
\end{equation}
\end{definition}
\section{Divergence free and rotational free space}

\begin{definition}[Divergence]
 The divergence of a vector consists of
 \begin{equation}
  \vect{\nabla} \cdot \vect{v}.
  \end{equation}
\end{definition}

\begin{definition}[Curl]
 The curl of vector consists of
 \begin{equation}
  \vect{\nabla} \times \vect{v}.
 \end{equation}
\end{definition}

\begin{definition}[Gradient]
 The gradient of a scalar consists of
 \begin{equation}
  \vect{\nabla} p.
 \end{equation}
\end{definition}

\begin{property}
 The gradient is curl free:
 \begin{equation}
  \vect{\nabla}\times \vect{\nabla} p=\vect{0}.
 \end{equation}
\end{property}

\begin{property}
The curl is divergence free:
 \begin{equation}
  \vect{\nabla}\cdot \vect{\nabla} \times \vect{v}=0.
 \end{equation}
\end{property}

\section{Projection}
\label{introduction:projection}
\begin{property}[Projection of a divergence free space]
 Every vector field can be projected onto a divergence free space without changing its curl with the following change:
\begin{align}
 \vect{\nabla}\vect{v_{new}}&=\vect{v}-\vect{\nabla}p \label{introduction:ProjectionA},\\
 \Delta p&=\vect{\nabla} \cdot \vect{v} \label{introduction:ProjectionB}.
\end{align}
\end{property}
\begin{proof}
  Taking the divergence of equation (\ref{introduction:ProjectionA}) and using (\ref{introduction:ProjectionB})
  \begin{equation}
   \vect{\nabla}\cdot\vect{v_{new}}=\vect{\nabla}\cdot\vect{v}-\Delta p=\vect{\nabla}\cdot\vect{v}-\vect{\nabla}\cdot\vect{v}=0.
  \end{equation}
  Taking the curl of equation (\ref{introduction:ProjectionA}) and using (\ref{introduction:ProjectionB})
  \begin{equation}
   \vect{\nabla}\times\vect{v_{new}}=\vect{\nabla}\times\vect{v}-\vect{\nabla}\times\vect{\nabla} p=\vect{\nabla}\times\vect{v}.
  \end{equation}
  \end{proof}
  
   \begin{remark}
    We have a Poisson's equation to solve. With Dirichlet and Neumann boundary condition we have a unique solution (if we are not
    Neumann everywhere).	
   \end{remark}

  
  \begin{remark}
 This works as long as
 \begin{equation}
  \Delta=\vect{\nabla}\cdot \vect{\nabla}.\label{introduction:ProjectionDelta}
 \end{equation}
 So this can also be used in the discretized case and will be exact as long as (\ref{introduction:ProjectionDelta}) is respected.
 \end{remark}


\begin{definition}[Linear projection operator]
\label{introduction:projectiondef}
We define the projection operator $P(\vect{v})$
  \begin{equation}
    P(\vect{v})=\vect{\nabla}\Delta^{-1}\vect{\nabla}\cdot \vect{v}.
  \end{equation}
  The projection of section \ref{introduction:projection} is then
  \begin{equation}
    \vect{v}_{new}=\vect{v}-P(\vect{v}).
  \end{equation}
\end{definition}
  In case we use boundary conditions that are not homogeneous Dirichlet or Neumann boundary condition,
  the projection operator is no longer linear, but affine.

  \begin{definition}[Affine projection operator]
  We denote the affine projection by $P(\vect{v},\vect{T})=A\vect{v}+\vect{T}$, where $A$ is a matrice and $\vect{T}$ a vector.
  
  The Poisson equation is given by
  \begin{equation}
   \Delta p=\vect{\nabla} \cdot \vect{v}.
  \end{equation}
This can be rewritten for Dirchlet or Neumann boundary conditions by
\begin{equation}
 B p=\vect{\nabla} \cdot \vect{v}-\vect{c},
\end{equation}
where $B$ is a matrix form of $\Delta$ and $\vect{c}$ is given by the boundary conditions.
$p$ is then given by
\begin{equation}
 p=B^{-1} \vect{\nabla} \cdot \vect{v}-B^{-1}\vect{c},
\end{equation}
and the projection is
\begin{equation}
 P(\vect{v},\vect{T})=-\vect{\nabla} p=-\vect{\nabla} B^{-1}\vect{\nabla} \cdot \vect{v}+\vect{\nabla} B^{-1}\vect{c}.
\end{equation}
This can be rewritten as
\begin{align}
P(\vect{v},\vect{T})&=A\vect{v}+\vect{T},\\
A&=\vect{\nabla} B^{-1}\vect{\nabla} \cdot,\\
\vect{T}&=\vect{\nabla} B^{-1}\vect{c}.
\end{align}
  
  \end{definition}
  
  The important change is that the projection of a divergence free vector field is no more 0,
  \begin{equation}
  P(\vect{v},\vect{T})=A\vect{v}+\vect{T}=0+\vect{T}=\vect{T}.
  \end{equation}



