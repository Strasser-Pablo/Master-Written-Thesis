\chapter{Introduction}
\minitoc
\section{Introduction}

\section{Notation and some Basic Math Property}

\subsection{Vector}

Vector with the dimension of physical space (2d or 3d) are written $\vect{v}$.

We only use Cartesian coordinate system.

We indicate a given component of a vector with indices. $\vect{v}_1$ indices first component.

\subsection{Differential operator}

\subsubsection{Component by component}

Derivative is written $\partial_i$ where $i$ is the spatial component with witch we derivate.
\begin{equation}
 \partial_i v(\vect{x})=\frac{\partial v(\vect{x})}{\partial \vect{x}_{i}}
\end{equation}

\subsubsection{Nabla}

\begin{definition}
 In Cartesian coordinate, the $\vect{\nabla}$ is a vector that cannot commute.
 It's expression is given by:
 \begin{equation}
 \vect{\nabla}_{i}=\partial_{i}
 \end{equation}
\end{definition}

With the $\vect{\nabla}$ we can form the following expression.
With $n$ the dimension of the vector:
\begin{equation}
 \vect{\nabla}p=\begin{pmatrix}\partial_1 p\\ \vdots \\ \partial_n p\end{pmatrix}
\end{equation}
\begin{equation}
 \vect{\nabla}\cdot \vect{v}=\sum_{i=1}^{n}\partial_i \vect{v}_i
\end{equation}
\begin{equation}
 \left((\vect{v}\cdot\vect{\nabla})\vect{v}\right)_{i}=\sum_{j=1}^{n}v_{j}\partial_{j} v_{i}
\end{equation}

\subsubsection{Laplacian}

\begin{definition}
The Laplacian $\Delta$ consist of a scalar operator witch can be seen as:
\begin{equation}
\Delta=\vect{\nabla}\cdot \vect{\nabla}=\sum_{i=1}^{n} \partial_{i}\partial_{i}
\end{equation}
\end{definition}

\begin{definition}
A vector version of the Laplacian consist to apply the Laplacian to all component of the vector.
We note the vector version of the Laplacian $\vect{\Delta}$
\end{definition}
\subsection{Divergence Free and Rotational Free Space}

\subsubsection{Definition}

\begin{definition}
 The Divergence of a vector consist of:
 \begin{equation}
  \vect{\nabla} \cdot \vect{v}
  \end{equation}
\end{definition}

\begin{definition}
 The Rotational of a vector consist of:
 \begin{equation}
  \vect{\nabla} \times \vect{v}
 \end{equation}
\end{definition}

\begin{definition}
 The Gradient of a scalar consist of:
 \begin{equation}
  \vect{\nabla} p
 \end{equation}
\end{definition}

\subsubsection{Property}

\begin{property}
 The divergence of a vector is Rotational Free.
 \begin{equation}
  \vect{\nabla}\times \vect{\nabla} p=\vect{0}
 \end{equation}
\end{property}

\begin{property}
The gradient of a vector is Divergence Free.
 \begin{equation}
  \vect{\nabla}\cdot \vect{\nabla} \vect{v}=0
 \end{equation}
\end{property}
\subsubsection{Projection}

\label{introduction:projection}
\begin{property}
 Every vector can be projected to a divergence free space without changing it's rotational with the following change:
\begin{align}
 \vect{v_{new}}&=v-\vect{\nabla}p \label{introduction:ProjectionA}\\
 \Delta p&=\vect{\nabla} \cdot \vect{v} \label{introduction:ProjectionB}
\end{align}
\end{property}
\begin{proof}
  Taking the divergence of equation \ref{introduction:ProjectionA} and using \ref{introduction:ProjectionB}
  \begin{equation}
   \vect{v_{new}}=\vect{\nabla}\cdot\vect{v}-\Delta p=\vect{\nabla}\cdot\vect{v}-\vect{\nabla}\cdot\vect{v}=0
  \end{equation}
  \begin{rem}
 This work as long as:
 \begin{equation}
  \Delta=\vect{\nabla}\cdot \vect{\nabla}\label{introduction:ProjectionDelta}
 \end{equation}
 So this can be used in discretized case and will be exact as long as \ref{introduction:ProjectionDelta} is respected.
 \end{rem}
\end{proof}

